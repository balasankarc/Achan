\section{അടുക്കളനൃത്തത്തിനു ഒരു പാട്ട്}
\obeylines
\noindent
ഇന്നാണ്‌ ഉണ്ണിയ്ക്കു പുറന്നാള്‌
ഇന്നാണ്‌ ഉണ്ണിയ്ക്കു പുറന്നാള്‌
എല്ലാരും എത്തുന്ന സുദിനം
അമ്മാത്തുന്നെല്ലാരും അച്ചോളുമാരെല്ലാം
അമ്മാവനമ്മായി നേരത്തെയും
ഏറ്റരി വെച്ചുള്ളതേവർക്കു നേദ്യം-പിന്നെ
എല്ലാർക്കും നല്ലൊരു സദ്യ തന്നെ
വടക്കിനിയടുക്കളയടിച്ചുതളി
പാത്രങ്ങളൊക്കെ മോറിയിടൂ
എന്താണു രധേ വിഭവങ്ങള്‌
അയ്യോയെന്റോപ്പോളെ അമ്മയെന്ത്യേ?
മുത്തശ്ശിയമ്മയും പേരശ്ശിയമ്മയും
 ചേർന്നുവന്നെത്തുന്നു കുഞ്ഞാത്തോലെ
“കുട്ടാ, കുട്ടാ, വന്നോളൂ പട്ടുകോണകം ഉടുത്തോളൂ
കണ്മഷിചെപ്പെന്ത്യേ മൂശേട്ടേ-കുട്ടനെ
പൊട്ടുതൊടീയ്ക്ക നീ അമ്മുക്കുട്ടി”
അമ്മ വന്നെത്തുന്നു വേഗത്തിലായ്
“ദേഹണ്ണമെല്ലാമെ വേഗമാവാം”
പച്ചടി കിച്ചടി ഉപ്പേരി കാളൻ
തോരനുമവിയൽ പരിപ്പുകറി
പാല്പ്പായസം നല്ലടപ്രഥമൻ
ഇഞ്ചിക്കറിയുമ ക്കാട്ടുകൂട്ടാൻ
കായയും നല്ല പടവലങ്ങ
ചേനയുമോലനു കുമ്പളങ്ങാ
എല്ലാമെടുത്തോളൂ നാണിപ്പെണ്ണേ
കുട്ടികളെല്ലാരും ഒത്തുകൂടി
ശീഘ്രം നുറുക്കേണമൊക്കെത്തിനും
സ്ക്വയറായിട്ടാണോയീയോലൻ മമ്മീ
ടെക്സാസു വല്ല്യഫൻ കൊച്ചുമോള്‌
ലോങ്ങായിവേണമവിയലിന്ന്
മോസ്കോയിലോപ്പോളിൻ ഉണ്ണിനങ്ങ
റിങ്ങായിട്ടരിയണം പാവയ്ക്കായെ
വറുത്തുപൊടിച്ചൊരുകിച്ചടിയ്ക്കായ്
കുന്നംകുളത്തുള്ളകുഞ്ഞഫന്റെ
പ്ളസ് ടൂവിലുള്ളോരുക്ടാവുചൊല്ലി
പാത്രങ്ങളൊക്കെയെടുത്തുവച്ചു
നിലകാതും ഓട്ടുരുളി കല്ച്ചട്ടിയും
അനിയത്തി ചൊല്ലുന്നു വെപ്രാളത്തിൽ
ഗ്യാസില്ല ഗോവിന്ദൻ വന്നതില്ലാ
സദ്യയും ഗോവിന്ദാ എന്നാവുമോ?
പവർ ഇന്നു ഷട്ഡൌൺ കൃത്യമായി
വിറകുണ്ടേലെടുത്തോളൂ നാണിപ്പെണ്ണേ
   നാണിപ്പെണ്ണു വന്നൂവേഗം
   ചുള്ളിക്കമ്പുകൾതിരുകിയടുപ്പിൽ
തീനാളങ്ങളയുർന്നുപൊങ്ങീ
അമ്മയും ചെറിയമ്മയും പേരമ്മയുംകൈപ്പുണ്യവും
കാളനോലനതിരസം ബഹുരുചികരം
കറിയൊക്കെയും
താളുതകരകളൊന്നുമല്ല ഊട്ടിവെജിറ്റബിൾസ്
മീൽസുഗ്രാന്റു മതായിടും
നല്ലൂണുരസികനുമായിടും

ഭാഗം രണ്ട്

ഊണായാൽ ഇല വേണം
വിഭവങ്ങളേറെവേണം
ഇലയ്ക്കതിന്നൊത്തതായവലിപ്പം വേണം
കുടിയ്ക്കുവാൻ വെള്ളം വേണം
വെള്ളത്തിൽ ചുക്കും വേണം
കുടിയ്ക്കുവാൻ പാകത്തിനേചൂടു പാടുള്ളൂ
സദ്യയായാൽ പ്രഥമൻ വേണം
പായസങ്ങൾ രണ്ടു വേണം
നാലുതരം ഉപ്പേരിയും വറുത്തു വേണം
സാമ്പാറിന്നുകായം വേണം
എരിശ്ശേരിയ്ക്കു വറ വേണം
കാളൻ വെച്ചു കുറുക്കിയാൽ കായയും വേണം
പുളിയിഞ്ചി എരിയൊല്ലാ
കടുമാങ്ങായെരിയണം
ഓലനായാൽ പുളി പോക്കാൻ ശക്തിയും വേണം
ചെത്തു മാങ്ങാക്കറിനല്ലൂ
നാരങ്ങായും വെള്ള നല്ലൂ
ഉപ്പുമാങ്ങാക്കറി വെയ്ക്കാൻ കർക്ക്ടം നല്ലൂ
എരിശ്ശേരി ചക്ക നല്ലൂ
പച്ചടിയ്ക്കു പഴമാങ്ങാ
പായസത്തിൻ ചെടിപ്പിനു നാരങ്ങാ നല്ലൂ
മോരുകൂട്ടി ഉണ്ണുവാനായ്
ചോറു ലേശം വിളമ്പേണം
ഉണ്ടു തീർന്നാൽ ഞാലിപ്പൂവൻ പഴവും കൊള്ളാം
ദേഹണ്ഡിയ്ക്കാനറിയണം
വിളമ്പാനുമതുപോലെ
സ്വാദറിഞ്ഞു സദ്യയുണ്ണാൻ കഴിവും വേണം
