\chapter{ഏട്ടാ....ഏട്ടാ...അനിയാ...അനിയാ..}
\obeylines
കർക്കടകത്തിലെ ഒരു ഞായറാഴ്ച. കുട്ടികളുമായി അമ്പലത്തിൽ പോയി തൊഴീൽ കഴിഞ്ഞു മതില്ക്കു പുറത്തിറങ്ങി. ആനപ്പന്തലിൽ ഏട്ടനും അനിയനും ഓട്ടം തന്നെ ഓട്ടം. സ്ക്കൂളില്ലാത്തതും ഓടിക്കളിയ്ക്കാൻ സ്ഥലം കിട്ടിയതും ആഘോഷിയ്ക്കുകയാണവർ. കുറേ ഓടിത്തളർന്നപ്പോൾ തിരികെപ്പോകുവാൻ സമ്മതിച്ചു. എല്ലാവരും കൂടി വീടെത്തി. എത്തിയ ഉടനെ രണ്ടുപേരും ധാരാളം വെള്ളം കുടിച്ചു. ഉടനെതന്നെ ജ്യേഷ്ഠൻ പോക്കറ്റിൽനിന്നും ഒരു പടം എടുത്തുപൊക്കിപ്പിടിച്ചു. അമ്പലത്തിൽ വെച്ചു എപ്പോഴോ സ്വാമിസ്സാർ കൊടുത്ത ശ്രീരാമന്റെ പടം. പടം കണ്ടപ്പോൾ അനിയന്‌ അതു കിട്ടണമെന്നായി. ഏട്ടനായ അപ്പു കൊടുത്തില്ല.അനിയൻ പിടിച്ചുവാങ്ങാൻ ശ്രമിച്ചു. ഏട്ടൻ അതും കൊണ്ടോടി. അനിയൻ പുറകെ ഓടി. രണ്ടുപേരും വീണ്ടും ഓട്ടം തന്നെ.
“അപ്വേ, നീ അന്യേനെ ഇങ്ങനെ ഓടിയ്ക്കാതെ” അമ്മ വിളിച്ചുപറഞ്ഞു.
അവരുണ്ടോ ഓട്ടം നിറുത്തുന്നു. ഇനി എന്താചെയ്ക? അമ്മ അവിടെയൊക്കെ നോക്കി. ഒരു ശ്രീകൃഷ്ണന്റെ പടം കിട്ടി. അതു അനിയന്റെ നേരേ നീട്ടി. അയാൾക്കതു പോരാ. ഏട്ടന്റെ കൈവശം ഉള്ള പടം തന്നെ വേണം. ഏട്ടൻ കൊടുക്കുകയുമില്ല. അനിയനു വേണം താനും. കുട്ടികൾക്കു വാശി കൂടിവന്നു. കുട്ട്യോൾടെ അച്ഛൻ എത്തിയിട്ടില്ല. ഇനി എന്താ വേണ്ടേ എന്നുവിചാരിച്ചു നില്ക്കുമ്പോൾ അതാ വാസ്വഫൻ പടികടന്നു വരുന്നു. ആശ്വാസമായി.
“കുട്ടികളേ,ആരാ വരുന്നതെന്നു നോക്കിയ്ക്കേ” അമ്മ വിളിച്ചുപറഞ്ഞു.
“വാസ്വഫൻ” രണ്ടുപേരും ഒരുമിച്ചു വിളിച്ചുപറഞ്ഞു.
വാസ്വഫനെ രണ്ടുപേർക്കും ഇഷ്ടമാണ്‌. വല്ല്യ വല്ല്യ കാര്യങ്ങൾ അഛനോടു സംസാരിയ്ക്കും. പലഹാരത്തിന്റെ കാര്യങ്ങൾ അമ്മയോടു സംസാരിയ്ക്കും. കുട്ടികൾക്കു കപ്പലണ്ടി കൊടുക്കും, കഥകൾ പറഞ്ഞുതരും. രണ്ടുപേരും തത്ക്കാലം വാശിയെ കാശിയ്ക്കു വിട്ടിട്ട് വാസ്വഫനടുത്തേയ്ക്കെത്തി.
“എന്താ അപ്പ്വേ കയ്യിലെ ചിത്രം? ശ്രീരാമനാ?ആയ്! രാമായണമാസം അല്ലേ? പണ്ടൊക്കെ ഇടവം മിഥുനം കർക്കടകം ചിങ്ങം എന്നൊക്കെയാ പറയാറ്‌. ഇപ്പോൾ അങ്ങനെയല്ല. കർക്കടകം ഇല്ല. പകരം രാമായണമാസം എന്നാപറയുക. മിഥുനത്തിലെ പെരുമഴയ്ക്കുശേഷം വരുന്ന കള്ളക്കർക്കടകം ഒരു പഞ്ഞമാസം തന്നെ. എത്ര ധനവാനും ഒരു ഇടിച്ചിൽ ഉണ്ടാവും. ഒന്നും നട്ടാൽ മുളയ്ക്കാത്ത, ഒന്നും ചെയ്യാൻ പറ്റാത്ത കാലം.
“കർക്കടമാസം കഴിയും വരേയ്ക്കിനി-
ക്കഞ്ഞിയാണുണ്ണി നിനക്കിഷ്ടമാവുമോ?”
എന്നാണ്‌ ഉണ്ണിയോട് അമ്മ ചോദിക്കുന്നത്. ഏതായാലും ഇപ്പോൾ മറ്റൊന്നുമില്ലെങ്കിലും കർക്കടകമാസതിൽ രാമായണം വായന നടക്കുന്നുണ്ട്. അതുകൊണ്ടാവാം ഇപ്പോൾ ദാരിദ്ര്യവും അല്പം കുറവുണ്ട്”.
അപ്പോഴേയ്ക്കും അമ്മ സംഭാരവുമായി വന്നു.അഫൻ സംഭാരം കുടിച്ചുകഴിഞ്ഞപ്പോൾ “ഇനി കഥ, ഇനി കഥ” എന്നുപറഞ്ഞു കുട്ടികൾ തിടുക്കം കൂട്ടി.
“കഥ മാത്രമല്ല കപ്പലണ്ടിയുമാവാം. നിങ്ങളുടെ ഓട്ടവും ബഹളവും കണ്ട് അതിന്റെ കാര്യം വിട്ടുപോയി”. ഓരോപൊതി കപ്പലണ്ടി രണ്ടു പേർക്കും കൊടുത്തു. എന്നിട്ട് പറഞ്ഞുതുടങ്ങി
“എന്നും രാമായണം വായിയ്ക്കുന്നുണ്ടല്ലോ. അതു ഭാഷാസ്വാധീനം ഉണ്ടാക്കും. ഇന്ന് നമുക്ക് രാമായണത്തിലെ ചിലരെ പരിചയപ്പെടാം. ആദ്യമേതന്നെ നിങ്ങളെപ്പോലെ ഓടിക്കളിയ്ക്കുന്ന, അല്ല, പറന്നു കളിയ്ക്കുന്ന ഒരേട്ടനും അനിയനും ആവട്ടെ. ഗരുഡൻ എന്നു കേട്ടിട്ടില്ലെ?”
“ഉവ്വ്, മഹാവിഷ്ണുവിന്റെ വാഹനം” - അപ്പു.
“ഗരുഡന്‌ ഒരേട്ടൻ ഉണ്ട്.അരുണൻ.”
“അരുണൻ എന്നൽ “അര”ണൻ എന്നല്ലെ?” - വീണ്ടും അപ്പു.
“അതെയതെ. അരയ്ക്കു കീഴ്ഭാഗം ഇല്ലാത്തയാൾ. സൂര്യരഥത്തിന്റെ തേരാളി.”
“സൂര്യന്റെ തേരിന്‌ ഏഴു കുതിരകളാണ്‌. അല്ലേ?”
“എന്നുപറഞ്ഞാൽ ഏഴു നിറങ്ങൾ ചേർന്നാണ് സൂര്യരശ്മിയുണ്ടായിരിയ്ക്കുന്നത് എന്നാണർത്ഥം”.
“മാനത്തെ മഴവില്ലിനേഴുനിറം
മനസ്സിന്റെ മാരി വില്ലിനേഴല്ലെഴുനൂറ്‌” അപ്പു ഒരു പാട്ടു പാടി
“മഴവില്ലിനും ഏഴു നിറമല്ല. എത്രയെന്നു പറയാൻ പറ്റുകയില്ല. അതൊക്കെ പ്രകാശത്തിന്റെ സ്വഭാവത്തെപ്പറ്റി കൂടുതൽ പഠിയ്ക്കുമ്പോൾ മനസ്സിലാകും. ഏതായാലും സൂര്യന്റെ ഡ്രൈവറായ അരുണൻ വിവാഹം കഴിച്ചു. ശ്യേനി അഥവാ മഹാശ്വേത”.
“അരുണം ചുവപ്പും ശ്വേതം വെളുപ്പും” കഴിഞ്ഞ ദിവസം ക്ലാസ്സിൽ രക്തത്തെപ്പറ്റി പഠിച്ചതോർത്ത് അപ്പു പറഞ്ഞു.
അഫൻ തുടർന്നു “അരുണനും ശ്യേനിക്കും രണ്ടു മക്കൾ. രണ്ടു കഴുകന്മാർ. ഏട്ടന്റെ പേര്‌ സമ്പാതി. അനുജൻ ജടായുസ്സ് അഥവാ ജടായു.
\begin{center}
അരുണപ്രിയയാം ശ്യേനി
പെറ്റൂ രണ്ടു കുമാരരെ
സമ്പാതിയെന്നു വൻപേറും
ജടായുസ്സെന്നുമങ്ങനെ
\end{center}
കുട്ടിക്കാലത്തു ഇവർ ഒന്നു പറന്നു കളിച്ചു. ആർക്കാണു കൂടുതൽ ഉയരത്തിൽ പറക്കാൻ കഴിയുന്നതെന്നറിയാൻ വാശിയോടെ മേല്പ്പോട്ടു പറന്നുയർന്നു. കൂടുതൽ ഉയരത്തിലേയ്ക്കു പോയ അനുജൻ ജടായുവിനു സൂര്യന്റെ ചൂടേറ്റ് ഒരു തളർച്ച അനുഭവപ്പെട്ടു. ഉടൻ തന്നെ ഏട്ടനായ സമ്പാതി ജടായുവിനു മുകളിൽ ഉയർന്നുപൊങ്ങി ചിറകു വിരിച്ചു പറന്നുനിന്നു തണലേകി. പക്ഷേ, എന്തു സംഭവിച്ചു? സൂര്യന്റെ ചൂടേറ്റു സമ്പാതിയുടെ ചിറകുകൾ കരിഞ്ഞുപോയി. സമ്പാതി താഴേക്കു വീണു, വിന്ധ്യപർവതത്തിൽ നിശാകരമുനിയുടെ ആശ്രമത്തിൽ. ശിഷ്ടകാലം മുനി ആഹാരം നല്കി സമ്പാതിയെ രക്ഷിച്ചു. സമ്പാതിയും ജടായുവും സീതാന്വേഷണത്തിൽ ശ്രീരാമനു സഹായികളായ കഥകൾ രാമായണത്തിൽ പറയുന്നുണ്ട്.
“ഓടിക്കളിക്കുമ്പോൾ അനിയൻ വീഴാതെ നോക്കേണ്ടതു ഏട്ടനാണെന്നാണ്‌ അഫൻ പറഞ്ഞുവരുന്നത്, കേട്ടോ അപ്പൂ” കഥ കേട്ടുകൊണ്ടു വന്ന അഛൻ പറഞ്ഞു.
“ഏട്ടൻ അനിയനെ രക്ഷിക്കുന്ന കാര്യമേ ഉള്ളൂ? അനിയനു എന്തും ആകാം,അല്ലേ?”അപ്പു പരിഭവിച്ചു
വാസ്വഫൻ തുടർന്നു
“രാമായണത്തിൽ തന്നെയുണ്ടല്ലോ നാലു ജ്യേഷ്ഠാനുജന്മാർ. നീണ്ട പതിന്നാലു വർഷം കൊടുങ്കാട്ടിലും മലയിലുമായി രാക്ഷസന്മാരുടെയും വന്യമൃഗങ്ങളുടെയുമിടയിൽ രാമേട്ടന്റെ സഹായിയായി ഒരനുജൻ ലക്ഷ്മണൻ. സ്വന്തം അമ്മയെയും ഭാര്യയേയും കൊട്ടാരത്തിൽ വിട്ട് തനിക്കുവേണ്ടിയല്ല, ഏട്ടനു വേണ്ടി ഇത്ര കഷ്ടപ്പാടുകൾ സഹിച്ചു ജീവിച്ച മറ്റൊരനുജൻ പുരാണത്തിലെങ്ങുമില്ല”.
“സമ്പാതി എന്ന ഏട്ടൻ അനിയനെ രക്ഷിച്ചതിലും എത്രയധികമാണ്‌ ലക്ഷ്മണൻ എന്ന അനുജൻ ചെയ്തിരിക്കുന്നത്. കൂയ്! കൂയ്! അനിയനാണ്‌ കൂടുതൽ മഹാൻ” അനിയൻ തുള്ളിച്ചാടി.
“കഥ തീർന്നില്ലല്ലോ” വാസ്വഫൻ ഇടപെട്ടു “തനിയ്ക്കവകാശപ്പെട്ട രാജപദവിയും കൊട്ടാരങ്ങളും ഭരതനു വേണ്ടി ഉപേക്ഷിച്ചിട്ടാണല്ലൊ ശ്രീരാമൻ കാട്ടിലേയ്ക്കു പോയത്”
“ഇപ്പോഴോ” അപ്പുവിനു കുറച്ചുകൂടി ഗമയായി.
“അതു വേറെ അനിയനല്ലേ” അനിയനും വിട്ടുകൊടുക്കാൻ തയ്യാറല്ല.
വാസ്വഫൻ ഇടപെട്ടു, “അവർ നാലു ജ്യേഷ്ഠാനുജന്മാരും പരസ്പരം സ്നേഹിച്ചും ബഹുമാനിച്ചുമാണ്‌ കഴിഞ്ഞിരുന്നത്. അത് രാമായണം കഥ മുഴുവൻ വായിച്ചാൽ മനസ്സിലാകും. അരും ആരെയുംകാൾ കേമനല്ല. ഇനിയുമുണ്ട് അനവധി ഏട്ടന്മാരും അനവധി അനിയന്മാരും
രാമാദികൾ നാലു സഹോദരന്മാരെപ്പോലെ രാവണാദികൾ നാലു സഹോദരങ്ങൾ. വൈശ്രവണൻ, രാവണൻ, കുംഭകർണൻ, വിഭീഷണൻ. ഇവരെല്ലാവരും വിശ്രവസ്സ് എന്ന മഹർഷിയുടെ പുത്രന്മാരാണ്‌. ഇവരിൽ വൈശ്രവണൻ യക്ഷന്മാരുടെ രാജാവും രാവണൻ രാക്ഷസന്മാരുടെ രാജാവും. തെക്കേ സമുദ്രത്തിൽ ത്രികൂടപർവതത്തിനു മുകളിൽ ഇന്ദ്രന്റെ രാജധാനിയെ വെല്ലുന്ന ഒരു നഗരം വിശ്വകർമാവു രാക്ഷസന്മാർക്കുവേണ്ടി ഉണ്ടാക്കിക്കൊടുത്തിരുന്നു. അവർ മഹാവിഷ്ണുവിനോടുണ്ടായ യുദ്ധത്തിൽ പരാജിതരായി പാതാളത്തിൽ പോയി ഒളിച്ചു പാർത്തു വരികയായിരുന്നു. അങ്ങനെ ശൂന്യമായിക്കിടന്ന ലങ്കയിൽ വിശ്രവസ്സിന്റെ നിർദ്ദേശപ്രകാരം വൈശ്രവണനെന്ന കുബേരൻ താമസമാക്കി. ലങ്കയെ സംരക്ഷിച്ചു ഐശ്വര്യം നിലനിർത്തി. സമുദ്രത്തിനു നടുവിലുള്ളദ്വീപിൽ നിന്നു പുറത്തേയ്ക്ക് പോകാൻ കുബേരനു മൂന്നു വിമാനങ്ങൾ ഉണ്ടായിരുന്നു. അതിൽ ഏറ്റവും കേമം ഭൂമിയിൽ ഒരിടത്തു നിന്നു മറ്റൊരിടത്തേക്ക്പോകുവാൻ ഉപയോഗിച്ചിരുന്ന പുഷ്പകവിമാനമാണ്‌. മറ്റു വിമാനങ്ങൾ ഭൂമിയിൽനിന്നു പുറത്തേക്കു പോകുവാനുള്ളതാണ്‌.
ഇങ്ങനെയിരിയ്ക്കുന്ന കാലത്താണ്‌ വിശ്രവസ്സിനു മറ്റു മൂന്നുപുത്രന്മാരുണ്ടാകുന്നത്. അവർ ബ്രഹ്മദേവനെ തപസ്സു ചെയ്ത് വലിയ വരങ്ങളൊക്കെ സമ്പാദിച്ചു തിരിച്ചു വന്നപ്പോൾ അവരുടെ അമ്മാത്തുമുത്തശ്ശനായ സുമാലിയും കൂട്ടരും രാവണനെ രാക്ഷസവംശത്തിന്റെ അധിപനായി അവരോധിച്ചു സ്വീകരിയ്ക്കുകയും രാക്ഷസന്മാർക്കവകാശപ്പെട്ട ലങ്ക കുബേരനിൽനിന്നു തിരിച്ചുപിടിയ്ക്കാൻ ആവശ്യപ്പെടുകയും ചെയ്തു. എന്നാൽ ജ്യേഷ്ഠനോട് അവിടെനിന്നു ഇറങ്ങിപ്പോകാൻ പറയാനുള്ള മടി കൊണ്ടങ്ങനെ പറഞ്ഞില്ല. വീണ്ടും സുമാലിയുടെയും കൂട്ടരുടെയും നിർബ്ബന്ധം കൂടിവന്നപ്പോൾ രാക്ഷസന്മാർക്കവകാശപ്പെട്ട ലങ്ക അവർക്കു തിരികെ നല്കണമെന്നു വളരെ മാന്യമായി ആവശ്യപ്പെട്ടു. ഏട്ടനായ വൈശ്രവണനാകട്ടെ അഛന്റെ അഭിപ്രായവും നിർദ്ദേശവും അനുസരിച്ച് ലങ്ക അനിയനും കൂട്ടർക്കും ഒഴിഞ്ഞു കൊടുത്ത് അളകാപുരിയിലേയ്ക്കുപോയി. എത്ര ഭംഗിയായിട്ടാണു ജ്യേഷ്ഠാനുജന്മാർ പെരുമാറിയത് പിന്നീടങ്ങോട്ടു രാവണന്റെ സ്വഭാവം ആകെപ്പാടെ മാറിപ്പോയി.”
“ഇനിയുമില്ലെ, അഫാ, മറ്റൊരേട്ടനും അനിയനും? ഞങ്ങൾ അടികൂടുമ്പോൾ മുത്തശ്ശൻ പറയാറുണ്ട് ‘ബാലിസുഗ്രീവന്മാർ’ തുടങ്ങി എന്ന്” - അപ്പു
“ശരിയാണ് പരസ്പരസ്നേഹത്തിന്റെയും ബഹുമാനത്തിന്റെയും ഉദാഹരണമായിരുന്നു ബാലി സുഗ്രീവന്മാർ.”
“അതു രണ്ടു കുരങ്ങന്മാരല്ലേ? അപ്പോ അടികൂടും”അനുജൻ
“അല്ല അവർ ജന്മനാ കുരങ്ങന്മാരല്ല. ജടായുവിന്റെയും സമ്പാതിയുടെയും അഛനായ അരുണൻ ഇവരുടെ അമ്മയാണ്‌ ഒരു ദിവസം അസ്തമയത്തിനുശേഷം അരുണൻ സൂര്യരഥം ഷെഡ്ഡിൽ കയറ്റിയിട്ട് സ്ത്രീരൂപം ധരിച്ചു ഇന്ദ്രസഭയിൽ അപ്സരസ്സുകളുടെ നൃത്തം കാണാൻ പോയി. പുതിയ കാഴ്ചക്കാരിയെ ഇന്ദ്രനു വളരെയധികം ഇഷ്ടപ്പെട്ടതുകൊണ്ട് അവൾക്ക് ഇന്ദ്രനിൽനിന്നും ഒരു കുട്ടിയെ ലഭിച്ചു. ഇക്കാരണത്താൽ അരുണൻ തേരിലെത്താൻ താമസിച്ചു. സൂര്യന്റെ പ്രയാണം വൈകി. അരുണനോട് സൂര്യൻ ദേഷ്യപ്പെട്ടു. കാരണം അറിഞ്ഞപ്പോൾ ആദിത്യനും ആ സ്ത്രീരൂപം ഒന്നുകാണണമെന്നുതോന്നി. അതു സാധിച്ചുകൊടുത്തപ്പോൾ സൂര്യനിൽനിന്നും ഒരു കുട്ടിയെ ലഭിച്ചു. അങ്ങനെ അരുണനു, അല്ല, അരുണിയ്ക്ക്, രണ്ടു കുട്ടികളായി.”
“മഹാവിഷ്ണുവിന്റെ മോഹിനീരൂപത്തിൽക്കൂടി ശിവന്റെ പുത്രനായി ശാസ്താവ് ജനിച്ചതുപോലെ, അല്ലേ?” അപ്പു.
“അതെ, രണ്ടു കുട്ടികളെയും ഇന്ദ്രന്റെ നിർദ്ദേശപ്രകരം വളർത്തുവാനായി അഹല്യയെ ഏല്പ്പിച്ചു. ഒരു ദിവസം കുട്ടികളെ പുഴയിൽ കൊണ്ടു പോയി കുളിപ്പിച്ച് അഹല്യ രണ്ടെളിയിലുമായി എടുത്തുകൊണ്ടുവരുന്നതു കണ്ട് മഹർഷി ഗൗതമനു കോപം വന്നു. കോപം വന്നാൽ എന്താചെയ്ക? ശപിച്ചു. കുരങ്ങുകുട്ടികൾ തള്ളക്കുരങ്ങിനെ അള്ളിപ്പിടിചുകൊണ്ടുവരുന്നതുപോലെ കണ്ടിട്ടു കുട്ടികൾക്കു കുരങ്ങിന്റെ സ്വഭാവം ആകട്ടെ എന്നാണ്‌ ശപിച്ചത്. അന്നുമുതൽ അവർ മർക്കടന്മാരായി. ഇതറിഞ്ഞു ദേവേന്ദ്രൻ രണ്ടു കുട്ടികളെയും കൊണ്ടുപോയി നന്ദനവനത്തിൽ താമസിപ്പിച്ചു. അക്കാലത്തു കിഷ്കിന്ധ എന്ന രാജ്യം വാണിരുന്ന മർക്കടരാജാവായ ഋക്ഷരജസ്സ് കുട്ടികളില്ലാതെ ദു:ഖിച്ചിരുന്നു. മന്ത്രിയായ ജാംബവാന്റെ നിർദേശപ്രകാരം ദേവേന്ദ്രനോടു പ്രാർത്ഥിച്ചതനുസരിച്ച് ഈ രണ്ടു കുട്ടികളെയും ഋക്ഷരജസ്സിനു നല്കി. വാലിനു ഭംഗിയുള്ളതുകൊണ്ട് ഇന്ദ്രപുത്രനു ബാലി എന്നും കഴുത്തിന്റെ ഭംഗി കണ്ടു സൂര്യപുത്രനു സുഗ്രീവൻ എന്നും പേരു നല്കിയിരുന്നു. ഋക്ഷരജസ്സിന്റെ കാലശേഷം മൂത്തവനായ ബാലി ചക്രവർത്തിയായും സുഗ്രീവൻ ഇളയരാജാവായും മന്ത്രിമാരോടുകൂടി വാണരുളുകയായിരുന്നു. അപ്പോഴാണ്‌ മായാവി എന്ന അസുരനുമായി ബാലിയ്ക്കു യുദ്ധം ചെയ്യേണ്ടി വന്നത്. യുദ്ധം ചെയ്തു ചെയ്തു രണ്ടുപേരും ഒരു ഗുഹയിൽക്കയറി. രക്തം കാണുകയാണെങ്കിൽ ഗുഹയടച്ചുപൊയ്ക്കൊൾവാൻ ബാലി സുഗ്രീവനു നിർദ്ദേശം നല്കി. ഒരുവർഷം കാത്തുനിന്നിട്ടും ആരും പുറത്തു വന്നില്ല. പക്ഷേ രക്തത്തിന്റെ പ്രളയം വന്നപ്പോൾ സുഗ്രീവൻ ഗുഹയടച്ചുപോയി കിഷ്ക്കിന്ധയിലെ രാജാവായി. പക്ഷേ മരിച്ചതു മായാവിയായിരുന്നു. ബാലി പുറത്തിറങ്ങാൻ വന്നപ്പോൾ ഗുഹയടച്ചിരിയ്ക്കുന്നതുകണ്ടു ദേഷ്യത്തോടെ തള്ളിത്തുറന്നു സുഗ്രീവനുമായി വഴക്കിട്ടു രാജ്യം കൈക്കലാക്കി. അന്നു രാജാവായിരുന്ന സുഗ്രീവനു വേണമെങ്കിൽ ബാലിയെ വധിയ്ക്കാമായിരുന്നു. പക്ഷേ, ജ്യേഷ്ഠനായതുകൊണ്ടു സുഗ്രീവൻ അതു ചെയ്തില്ല. സുഗ്രീവൻ എന്തൊക്കെപ്പറഞ്ഞിട്ടും ബാലിയ്ക്കു വിശ്വാസമായില്ല. ബാലി പുറത്തിറങ്ങാതിരിയ്ക്കാൻ വേണ്ടിത്തന്നെയാണ്‌ഗുഹയടച്ചതെന്നു വിശ്വസിച്ചു. ആ വൈരം പിന്നീട് തുടരുകയും ചെയ്തു. പരസ്പരസ്നേഹത്തോടുകൂടി പെരുമാറിയിരുന്ന അവർക്കിടയിൽ ഉണ്ടായ തെറ്റിദ്ധാരണയാണ് അകൽച്ചയ്ക്കു കാരണമായത്.
ഇങ്ങനെ നോക്കിയാൽ അടിസ്ഥാനപരവും സുദൃഢവും ആയ സഹോദരസ്നേഹത്തെ വിശദമാക്കുന്നതാണ് രാമായണത്തിലെ പല കഥാഭാഗങ്ങളും. മഹാഭാരതത്തിലാണെങ്കിൽ രണ്ടു തായ്‌വഴികളിലെ സഹോദരന്മാർ തമ്മിൽ സ്വത്തിനു വേണ്ടി നടത്തുന്ന വഴക്കുകളാണ്. രാമായണസഹോദരന്മാർ തമ്മിൽ സ്വത്തു തർക്കം ഉണ്ടാവുന്നതേയില്ല.”
എല്ലാവരും വാസ്വഫൻ പറഞ്ഞതിനെപ്പറ്റി ഓർത്തുകൊണ്ടിരുന്നപ്പോൾ യാദൃഛികമായി റ്റി വി യിൽ കുഞ്ഞുണ്ണിമാഷും കുട്ട്യോളും എന്ന പരിപാടി താനേ തെളിഞ്ഞുവന്നു. മാഷ് കുട്ടികളെക്കൊണ്ടു ചൊല്ലിയ്ക്കുകയാണ്‌

\begin{minipage}{\textwidth}
ക ച ട ത പ ക ച ട ത പ
കുറച്ചെനിയ്ക്കേട്ടാ കദളിപ്പഴം
------------------------------
കൈപെനിയ്ക്കിഷ്ടമാണേറ്റമേട്ടാ”
\end{minipage}
