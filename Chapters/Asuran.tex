\chapter{അസുരൻ}
സ്കൂളിൽനിന്നുവന്ന ഉടനെ അപ്പു വിശേഷങ്ങൾ പറയാൻ തുടങ്ങി - "ഇന്നു ടീച്ചർ ഓണത്തിന്റെ കഥ പറഞ്ഞു തന്നു. മഹാബലി എന്ന അസുര ചക്രവർത്തിയെ സ്വീകരിയ്ക്കാനാണ്‌ നമ്മൾ ഓണം ആഘോഷിയ്ക്കുന്നത്. മനുഷ്യരുടെ ചക്രവർത്തിയല്ലെങ്കിൽ പിന്നെ എന്തിനാണ്‌ നമ്മൾ അസുരചക്രവർത്തിയെ എതിരേല്ക്കുന്നത്?.ആരാണച്ഛാ അസുരൻ?"\\
"സുരന്മാരല്ലാത്തവരാണ്‌ അസുരൻ" - അച്ഛൻ.\\
"സത്യം അല്ലാത്തതു അസത്യം ആയതു പോലെ ,അല്ലേ അഛാ?"\\
"അതെ, ഇഷ്ടമല്ലാത്തത് അനിഷ്ടം, സഭ്യമല്ലാത്തത് അസഭ്യം. നാമങ്ങളുടെയോ ക്രിയകളുടെയോ ഒക്കെ പുറകിൽ 'അ' ചേർത്താൽ വിപരീതാർത്ഥം ലഭിയ്ക്കുന്നു. ശക്തൻ അല്ലാതത് അശക്തൻ. ശുദ്ധം അല്ലാത്തത് അശുദ്ധം. ശ്രദ്ധയില്ലാത്തത് അശ്രദ്ധ"\\
"പൂയ് അല്ലാത്തത് അപ്പൂയ്" എന്നു പറഞ്ഞു കൊണ്ട് അനിയൻ അതിലെ വന്ന് ഓടിപ്പോയി.\\
"ഇപ്പോൾ ധാരാളമായി ഉപയോഗിയ്ക്കുന്ന ഒരു പദമാണല്ലോ ആറ്റം. ഒരു മൂലകത്തിന്റെ ഏറ്റവും ചെറിയ യൂണിറ്റ്. വിഭജിക്കാനാകാത്ത ഏറ്റവും ചെറുത് എന്ന അർത്ഥത്തിലാണ്‌ ആറ്റം എന്ന പേരു് കൊടുത്തത്.പക്ഷേ, ആറ്റത്തിനെ തകർത്തു തരിപ്പണമാക്കിക്കളഞ്ഞില്ലേ. ഇപ്പോൾ വിഭജിയ്ക്കനാകാത്ത ഏറ്റവും ചെറുത് ഏതെന്നു പറയാൻ ആവാത്ത അവസ്ഥയിലാണ്‌ ദൈവകണം വരെ എത്തിനില്ക്കുന്ന ഭൌതികശാസ്ത്രം. ഗ്രീക്കുഭാഷയിൽ ടോമോസ് എന്നുപറഞ്ഞാൽ പൊട്ടിക്കാവുന്നത് എന്നാണ്‌. പുറകിൽ 'അ' ചേർക്കുമ്പോൾ പൊട്ടിക്കാനാവാത്തത് എന്നാകും"\\
"ഠം എന്നു കേൾക്കുമ്പോൾതന്നെ എന്തോ പൊട്ടുന്നതുപോലെ തോന്നും" - അപ്പു.\\
"അപ്പോൾ ഠം അല്ലാത്തത് ആഠം അഥവാ ആറ്റം"\\
"ഗ്രീക്കുഭാഷയിലും നമ്മുടെ ഭാഷയിലെപ്പോലെയാണ്‌, അല്ലേ,അഛാ?"\\
"ശരിയാണ്‌.എല്ലാ ഭാഷയുടെയും മൂലം ഒന്നുതന്നെയെന്നു വിശ്വസിക്കേണ്ടിയിരിയ്ക്കുന്നു. സംസ്കൃതത്തിൽ ദുഹിതാവ് എന്നുപറഞ്ഞാൽ പുത്രി എന്നാണർത്ഥം. ഈ പദത്തിന്റെ അർത്ഥം പാൽ കറക്കുന്ന ആളെന്നേ ഉള്ളൂ. പണ്ട് പശുവിനെ കറന്നിരുന്നത് പുത്രികളായിരുന്നിരിക്കണം. ഇംഗ്ലീഷിലെ ഡാട്ടർ ഇതിനു സമാനമാണ്‌. അതാകട്ടെ ഗ്രീക്കുഭാഷയിലെ 'ഥുഗാതർ' എന്ന ശബ്ദത്തിന്റെ തദ്ഭവമാണ്‌. പാണിനീയ പ്രദ്യോതത്തിന്റെ കർത്താവായ ശ്രീമാൻ ഐ.സി ചാക്കോയുടെ അഭിപ്രായം ഇങ്ങനെയാണ്‌.
ലോകത്തിൽ പല ഭാഷകൾ ഉണ്ടായതിനെപ്പെറ്റി ഗ്രീക്കുപുരാണത്തിൽ ഒരു കഥയുണ്ട്. ഭൂമിയിലെ മനുഷ്യന്മാർ എല്ലാവരും കൂടി വളരെ വലിയ ഒരു കൊട്ടാരം പണിയാൻ തുടങ്ങി. അതു ഉയർന്നു സ്വർഗത്തോളം എത്താറായി. ഇതു കണ്ടപ്പോൾ ദേവന്മാർക്കസൂയ മൂത്തു. എല്ലാവരുടെയും ഭഷ പലതാകട്ടെ എന്നു ശപിച്ചു. ഒരാൾ പറയുന്നതു മറ്റൊരാൾക്കു മനസ്സിലാകാതെ വന്നപ്പോൾ കൊട്ടാരം പണി അവിടെ നിലച്ചു പോയി.\\
"ഏതായാലും നമ്മൾ കഥ കേട്ടതുകൊണ്ട് അസുരൻ രക്ഷപ്പെട്ടു" - അപ്പു.\\
"ഇല്ല, നമുക്കസുരനിലേക്ക് തിരിച്ചുവരാം. സുരൻ എന്നപദത്തിന്റെ നിരുക്തം നമുക്കൊന്നു നോക്കാം"\\
"എന്താണച്ഛാ നിരുക്തം എന്നുപറഞ്ഞാൽ?"\\
"നിർ നിശ്ചയേന ഉക്തം നിരുക്തം. നിശ്ചയമായി സംശയമില്ലാതെ അർത്ഥം വ്യക്തമാക്കുന്ന ഒരു രീതിയാണ്‌ നിരുക്തം. ഉല്പ്പത്തിയും കൂടി പരിഗണിയ്ക്കുമ്പോൾ ഒരു പദത്തിനു പല അർത്ഥങ്ങളും കിട്ടാം. സംസ്കൃതത്തിൽ, വിശേഷിച്ച് വൈദികസംസ്കൃതത്തിൽ ആണ്‌ ഇതിനു പ്രാധാന്യം. സുരൻ എന്നാൽ ശക്തിയുള്ളവൻ എന്നും നല്ലതു നൽകുന്നവൻ എന്നും നല്ലവണ്ണം ശോഭിയ്ക്കുന്നവൻ എന്നുമൊക്കെ അർത്ഥം പറയാം. ഇതൊക്കെയുള്ളവരാണ്‌ ദേവന്മാർ. മറ്റൊരു നിരുക്തമുണ്ട്.അത് ഏറെ തെറ്റിദ്ധരിയ്ക്കപ്പെട്ടിരികുന്നതുമാണു്. സമുദ്രത്തിൽനിന്നുണ്ടായ സുര(മദ്യം) ഇവർക്കുള്ളതുകൊണ്ട് സുരന്മാർ. സുര ഇല്ലാത്തവർ അസുരന്മാർ. അതായത് ദേവന്മാർ അല്ലാത്തവർ അസുരന്മാർ. സുരന്മാർക്കു വിരുദ്ധരായവർ, സുരന്മാരിൽ നിന്ന് അന്യരായവർ.ഏവർക്ക് സുര(മദ്യം)ലഭിച്ചില്ലയോ അവർ അസുരന്മാർ".\\
"അപ്പോൾ മദ്യപാനികൾ ദേവന്മാരും മദ്യപാനികളല്ലാത്തവർ അസുരന്മാരും. രസമായിരിയ്ക്കുന്നു. എനിക്ക് അസുരനായാൽ മതി" - അപ്പു.\\
നമ്മുടെ സങ്കല്പത്തിനു കടകവിരുദ്ധം.അല്ലേ?വാല്മീകിയ്ക്കും തെറ്റു പറ്റുമോ?\\
\begin{minipage}{\linewidth}
\begin{center}
സുരാ പ്രതിഗ്രഹാദ്ദേവാ:\\
സുരാ:ഇത്യഭിവിശ്രുതാ:\\
അപ്രതിഗ്രഹണാത്തസ്യാ\\
ദൈതേയാശ്ചാസുരാസ്തഥാ\\
\end{center}
\end{minipage}\\\\
മദ്യത്തെപ്പറ്റിയുള്ള നമ്മുടെ സങ്കല്പമാണ്‌ ഈ കുഴപ്പങ്ങൾക്കെല്ലാം കാരണം. സുര എന്ന മദ്യം എന്താണെന്നു നോക്കാം. 
\begin{center}
അബ്ധിജാ സുരാ ഏഷാം അസ്തി ഇതി സുരാഃ
\end{center}
സമുദ്രത്തിൽനിന്നുണ്ടായതാണ്‌ സുര. അത് മറ്റൊന്നുമല്ല, പാലാഴിമഥനത്തിൽ ധന്വന്തരീഹസ്തത്തിൽ കൂടി കിട്ടിയ അമൃതം തന്നെ. അമൃതം ദേവന്മാർക്കാണല്ലോ കിട്ടിയത്. ഇനി യോഗശാസ്ത്രപരമായി ആലോചിക്കാം. മദ്യം എന്തു നല്കുന്നു? ലഹരി. അതായത് അതിയായ ആനന്ദം. ഇതു പക്ഷേ മദ്യത്തിന്റെ വീര്യം നിലനില്ക്കുന്ന നേരത്തേയ്ക്കുമാത്രം. സുര എന്ന ആനന്ദം പരമമായ ,എന്നെന്നും നിലനില്ക്കുന്ന ആനന്ദമാണു്. യോഗസാധനയിൽ മൂലാധാരത്തിൽനിന്നു പുറപ്പെടുന്ന ശക്തി സഹസ്രാരത്തിലെത്തി ശിവശക്തി ഐക്യം പ്രാപിക്കുന്നു. അതായത് ജീവാത്മാവും പരമാത്മാവും ഒന്നാണെന്ന ബോധം ഉണ്ടാകുന്നു. പ്രകൃതിയും പുരുഷനും സമഞ്ജസമായി സമ്മേളിക്കുന്നു. ഒന്നായിത്തീരുന്നു. ഈ ദ്വൈതബോധം ബ്രഹ്മാനന്ദത്തെ ഉണ്ടാക്കുന്നു. ഇതാണ്‌ അമൃതം, ഇതാണ്‌ മദ്യം. ഇതനുഭവിക്കുന്നവരാണ്‌ ദേവന്മാർ അഥവാ സുരന്മാർ. അസുരന്മാർക്കിതില്ല.\\
അക്കിത്തം പറയുന്നുണ്ട്-
\begin{center}
\hspace{2em}മദ്യം പോലെ കുടിയ്ക്കുന്നൂ ഞാൻ\\
\hspace{2em}ഹൃദ്യം നാരായണനാമം\\
\end{center}
നാരായണനാമം സാധകം ചെയ്യുന്നതിലൂടെ ഭക്തി മൂത്ത് ഉന്മാദാവസ്ഥയിലാകുന്നു. അതായത് നാരായണനാമം മദ്യം പോലെയാണ്‌. അസുരന്മാർക്ക് യോഗസാധനയിലൂടെ ഈ ആനന്ദം ലഭിയ്ക്കുന്നില്ല. അസുരന്മാരിലും ദേവന്മാരിലും പല തരക്കാരുണ്ട്. മഹാബലി നല്ല അസുരനല്ലേ? രാവണൻ നല്ല രാക്ഷസനല്ലേ? ദേവേന്ദ്രൻ അത്ര നല്ലവനാണോ? ശനിദേവൻ എങ്ങനെയാ? സ്വഭാവത്തെ അടിസ്ഥാനമാക്കി ദേവന്മാരിലും അസുരന്മാരിലും ബ്രാഹ്മണ ക്ഷത്രിയ വൈശ്യ ശൂദ്രന്മാരുണ്ട്. ഇതു അവർ ഉപയോഗിയ്ക്കുന്ന ലഹരിവസ്തുക്കളെയും ആശ്രയിച്ചിരിക്കുന്നു. അസുരന്മാരിലെ ബ്രാഹ്മണനു്പാലും ക്ഷത്രിയനു് നെയ്യും വൈശ്യനു് തേനും ശൂദ്രനു് കള്ളും ലഹരിയുണ്ടാക്കുന്നു. അസുരന്മാരിലെ തന്നെ ഏറ്റവും മോശമായവർ ഉപയോഗിക്കുന്നതാണ്‌ കള്ള്. മനുഷ്യൻ ഉപയോഗിക്കുന്ന ലഹരി അസുരന്മാരെപ്പോലും നാണിപ്പിക്കും.\\
"അപ്പോൾ മനുഷ്യനെക്കാൾ നല്ലവർ അസുരന്മാരാണെന്നാണോ?"\\
അസുരധർമ്മപഞ്ചകം അനുസരിച്ചു ജീവിക്കുന്ന അസുരന്മാർ നല്ല അസുരന്മാരാണ്‌. കൈയൂക്ക്, മത്സരം, യുദ്ധം, നീതിശാസ്ത്രവിജ്ഞാനം, ശിവഭക്തി ഇവയാണാ അഞ്ചു കാര്യങ്ങൾ.
