\chapter{ഇത ഓം തത് സത്}
ഇപ്പോൾ ചാരിറ്റബിൾ പ്രവർത്തനങ്ങൾ ധാരാളം നടക്കുന്ന കാലമാണ്‌. ചാരിറ്റബിൾ സംഘടനകളും ധാരാളമുണ്ട്‌. പലതും മേനി നടിക്കാനും കള്ളത്തരങ്ങൾ മറയ്ക്കാനും വേണ്ടി മാത്രമാണ്‌. ഇത്ര കുട്ടികൾക്കു ഭക്ഷണം കൊടുത്തു, ഞാൻ ഇന്ന സംഘടനയിലെ അംഗമാണ്‌ ,ഇത്ര കുട്ടികൾക്കു വിദ്യാഭ്യാസ സഹായം നല്കി എന്നൊക്കെ പറഞ്ഞു നടക്കുന്നവരുണ്ട്‌. ചില സംഘടനകൾ ഇത്തരം പ്രവർത്തനങ്ങൾ പടം സഹിതം പത്രങ്ങളിൽ കൊടുക്കാറുണ്ട്‌. ഇവരിൽ പലരും വീട്ടിൽ ഒരു ഭിക്ഷക്കാരൻ വന്നാൽ ഒന്നും കൊടുക്കാതെ ഓടിച്ചു വിടുന്നവരാണ്‌. \par
എന്താണ്‌ ദാനധർമ്മം? ദാനധർമ്മത്തിന്റെ മൂർത്തിമദ്ഭാവങ്ങളാണ്‌ കർണ്ണൻ, രാജാ ശിബി, ആദിശങ്കരാചാര്യർ തുടങ്ങിയവർ. സ്വന്തം ജീവൻ പോലും അപകടത്തിലാകുമെന്നും യാചിച്ചു് വന്നിരിക്കുന്നതു യഥാർത്ഥ ബ്രഹ്മണനല്ലെന്നറിഞ്ഞിട്ടും കർണ്ണൻ തന്റെ കവചകുണ്ഡലങ്ങൾ അർത്ഥിക്കു ദാനം ചെയ്തു, സ്വയം മരണത്തിലേക്കു നടന്നടുത്തു.
\begin{center}
“ദാനധർമ്മത്തിലോ കർണ്ണന്റെ ചേച്ചി നീ”
\end{center}
വേടന്റെ അമ്പേറ്റു മടിയിൽ വന്നു വീണ പ്രാവിനെ രക്ഷിക്കുവാൻ ശിബിമഹാരാജാവ് സ്വന്തം ഹൃദയമാംസമല്ല തന്നെത്തന്നെയാണ്‌ വേടനു സമർപ്പിച്ചത്. അസുരന്മാരെത്തുരത്താൻ സ്വന്തം നട്ടെല്ലൂരിക്കൊടുത്ത ദധീചി മഹർഷി, കാപാലികന്റെ മോക്ഷപ്രാപ്തിക്കായി ബലിയർപ്പിക്കാൻ സ്വന്തം ശിരസ്സുതന്നെ വെട്ടിയെടുത്തുകൊള്ളുവാൻ അനുവാദം കൊടുത്ത ശങ്കരാചാര്യർ ഇങ്ങനെ എത്രയോ ഉദാഹരണങ്ങൾ. \par
ദാനത്തിനു അർഹതയാണ്‌ പ്രധാനം. വെറുതെ ആവശ്യമില്ലാത്തവർക്ക് നല്കിയതുകൊണ്ട് കാര്യമില്ല. ഇപ്പോൾ മിക്കവാറും എല്ലാ അമ്പലങ്ങളിലും പ്രസാദമൂട്ട് ഉണ്ട്. വൈഷ്ണവക്ഷേത്രങ്ങളിൽ തിരുവോണം, രോഹിണി, ശിവക്ഷേത്രങ്ങളിൽ തിരുവാതിര, അയ്യപ്പക്ഷേത്രങ്ങളിൽ ഉത്രം, ദേവീക്ഷേത്രങ്ങളിൽ കാർത്തിക. ഇങ്ങനെ ദേവന്റെ സ്വഭാവമനുസരിച്ച് ഓരോ ദിവസവും. ഈ ദിവസങ്ങളിൽ ക്ഷേത്രത്തിനു സമീപം താമസിക്കുന്നവർ ആരും തന്നെ ആഹാരം ഉണ്ടാക്കുകയില്ല. അത്രയും ലാഭം. എന്നു മാത്രമല്ല ഊണിന്റെ സമയമാകുമ്പോൾ അമ്പലത്തിൽ  ചെന്നു ‘ആരെങ്കിലും കഴിക്കണ്ടേ’ എന്നൊരു സൌമനസ്യഭാവത്തിൽ പ്രസാദം ഊട്ട് കഴിച്ചു പോരും. അതുപോലെയാണ്‌ ഭാഗവത സപ്താഹങ്ങൾ നടക്കുന്ന സ്ഥലങ്ങാളിലും. അവിടെ ഇതിലും ലാഭമാണ്‌. ഏഴുദിവസത്തേക്ക് സർവ്വ ചിലവുകളും സപ്താഹസ്ഥലത്തു നിന്നാണ്‌. രാവിലെ കാപ്പി, ഉച്ചയ്ക്ക് ഊണ്, വൈകുന്നേരം കാപ്പി. ഈ സമയങ്ങളിൽ മാത്രം ഹാളിൽ നിറച്ചും ആളുണ്ടായിരിക്കും. അവരുടെ ശ്രദ്ധ ഭാഗവതത്തിലല്ല. എന്നാൽ ചായപ്പാത്രം വന്നോ, ഊണിനു ഇല വച്ചോ എന്നൊക്കെയായിരിക്കും. ഇങ്ങനെയുള്ള അനർഹർക്ക് പ്രസാദം ഊട്ട് കൊടുത്താൽ ദേവൻ അനുഗ്രഹിയ്ക്കുകയല്ല, നിഗ്രഹിയ്ക്കുകയാകും ചെയ്യുക.\par
ഇത് ആവശ്യമില്ലാത്തവർക്കു കൊടുക്കുന്നതാണെങ്കിൽ ഇനി മറ്റൊരു കൂട്ടരുണ്ട് ആവശ്യമില്ലാത്തവ കൊടുക്കുന്നവർ. നമുക്കാവശ്യമില്ലാത്തവ ആർക്കെങ്കിലുമൊക്കെ കൊടുക്കും. പറ്റുമെങ്കിൽ നാലാൾ കാൺകെത്തന്നെ. ഇതും ദാനത്തിൽ പെടുകയില്ല. എന്നാൽ ഇതൊരു സദ്പ്രവൃത്തിയാണ്‌. വെറുതെ കളയുന്നതിനു പകരം അതുപയോഗിക്കുന്നവർക്ക് കൊടുത്തുവല്ലോ. ഇതു പോലെ മറ്റൊരു സദ്പ്രവൃത്തിയാണ് കല്യാണത്തിനും മറ്റും ക്ഷണിച്ച ആയിരക്കണക്കിനാളുകൾ വരായ്കയാൽ മിച്ചമുള്ള ആഹാരസാധനങ്ങൾ ബാലമന്ദിരങ്ങളിലും അഗതി മന്ദിരങ്ങളിലും മറ്റും കൊടുക്കുന്നത്. ഉപയോഗയോഗ്യമായവ ഉപയോഗിക്കാമല്ലോ.\par
പിന്നെ എന്താണ്‌ ദാനം? ഒരാൾക്ക് ഏറ്റവും അത്യാവശ്യമായ കാര്യം പ്രതിഫലേച്ഛ കൂടാതെ ചെയ്തുകൊടുക്കുമ്പോഴാണ്‌ അത് ദാനം ആകുന്നത്. അത് ദ്രവ്യമാകാം, ഉപദേശമാകാം, സാന്ത്വനമാകാം, അങ്ങനെ എന്തും. ദാനം കൊടുക്കുന്ന വസ്തുവിന്റെ വിലയല്ല ദാനത്തിന്റെ മാഹാത്മ്യത്തിനാധാരം. അത് ലഭിക്കുന്ന ആളുടെ ആവശ്യത്തിന്റെ തീക്ഷ്ണതയാണ്‌. ചിലർക്ക് ചില സമയത്ത് പണം കൊടുക്കുന്നതിനേക്കാൾ ശരിയായ ഒരു ദിശ കണിച്ചുകൊടുക്കുന്നതാവും ആവശ്യം. അപ്പോൾ അതാണ്‌ കൊടുക്കേണ്ടത്.
ദാനവും ദക്ഷിണയുമൊക്കെ കൊടുത്തുകഴിയുമ്പോൾ പറയുന്ന ഒരു കാര്യമുണ്ട് - \textbf{ഇത ഓം തത് സത്} - ഇത് അങ്ങയുടേതാണ്. ദാനത്തിനെ ദൈവികമാക്കുന്നതിനാണ്‌ ഓം ചേർക്കുന്നത്. കൊടുത്തുകഴിഞ്ഞാൽ അത് പൂർണ്ണമായും സമർപ്പിക്കണം. പിന്നെ അതിനെപ്പറ്റി മനസ്സിൽ പോലും വിചാരിക്കരുത്. ചിലർ ഇതുപോലെ എന്തെങ്കിലും സഹായം ചെയ്താൽ പാടിക്കൊണ്ടു നടന്നില്ലെങ്കിലും സഹായം സ്വീകരിച്ചവനെക്കാണുമ്പോൾ മനസ്സിലെങ്കിലും ഒരു അധീശത്വം ഭാവിക്കും. ഇങ്ങനെയൊക്കെയുള്ള മനോഭാവം വച്ചുകൊണ്ടിരുന്നാൽ ദാനം പൂർണ്ണമാവുകയില്ല. ദാനവസ്തുവിലെ പിടി ദാതാവ് വിടുന്നില്ല. അറിയാതെപോലും ദാനം കൊടുത്ത വസ്തു തിരിച്ചെടുക്കരുത്. നൃഗൻ എന്ന രാജാവിന്റെ കഥ കേട്ടിട്ടില്ലേ? ദാനം കൊടുത്ത പശു തിരികെ വന്നത് അറിയാതെ മറ്റു പശുക്കളോടൊപ്പം നൃഗൻ അതിനെയും മറ്റൊരാൾക്ക് ദാനം കൊടുത്തു. അതായത് ഒരു പശുവിനെ രണ്ടുപേർക്ക് ദാനം കൊടുത്തു. അവർ തമ്മിൽ വഴക്കായി രണ്ടുപേരും ദാനം ഉപേക്ഷിച്ചുപോയി. മരണകാലത്ത് യമധർമ്മൻ നൃഗന്‌ ഒരു ശിക്ഷ കൊടുത്തു. ഒരു ഓന്തായി അനേകകാലം ഒരു പൊട്ടക്കിണറ്റിൽ കിടക്കുക. അറിയാതെ വന്ന അപരാധത്തിനാണോ ഇത്രയും വലിയ ശിക്ഷ എന്നുചോദിച്ച രാജാവിനോട് “അല്ലതാങ്കൾ പ്രശസ്തിക്കു വേണ്ടിയാണ്‌ ദാനം ചെയ്തത്. ദാനം മേടിച്ചവരുടെ പുകഴ്ത്തലിനുവേണ്ടിയാണ് അങ്ങിതു ചെയ്തത്. രാജ്യത്ത് ഒരു നേരത്തെ ആഹാരത്തിനുപോലും വകയില്ലാതെ ധാരാളം പേരുണ്ടായിരുന്നു. എല്ലാപ്രജകളെയും ഒരുപോലെ വീക്ഷിച്ചില്ല എന്ന അപരാധത്തിന്റെ ഫലമാണ്‌.”എന്നാണ്‌ പറഞ്ഞത്. ഇവിടെയും യഥാർത്ഥ തെറ്റ് പ്രശസ്തിക്കുവേണ്ടി ദാനം ചെയ്തു എന്നതാണ്. ഇന്നുപലരും ചെയ്യുന്ന ദാനധർമ്മങ്ങൾ, ചാരിറ്റബിൾ പ്രവർത്തനങ്ങൾ പ്രശസ്തിക്കുവേണ്ടിമാത്രമല്ല വോട്ടിനും അധികാരം നിലനിർത്തുനത്തിനും കൂടിയാണ്‌.\par
ദാനം കൊടുത്ത മറ്റൊരാളുടെ കഥ നോക്കാം. മഹാബലി തന്റെ മുത്തശ്ശനായ പ്രഹ്ളാദനെക്കണ്ടപ്പോൾ പറഞ്ഞു ദിവസവും പതിനായിരം ബ്രാഹ്മണർക്ക് ദാനം നല്കുന്നുണ്ടെന്ന്. നിന്റെ ഭരണം മഹാമോശമാണെന്നാണ്‌ പ്രഹ്ലാദൻപറഞ്ഞത്. “ഇത്രയധികം ദരിദ്ര ബ്രാഹ്മണർ അവിടെയുണ്ടോ? നീയെന്താണ്‌ പ്രജകളുടെ ദാരിദ്ര്യം മാറ്റാൻ ശ്രമിക്കാത്തത്?” എന്നാണ്‌ ചോദിച്ചത്.
സ്വർണ്ണക്കീരിയുടെ കഥ കേട്ടിട്ടില്ലേ? ധർമ്മപുത്രർ നടത്തിയ രാജസൂയം യാഗത്തിൽ അനേകം ബ്രാഹ്മണർക്ക് ദാനം നല്കി. ശരീരത്തിൽ പാതിഭാഗം സ്വർണ്ണനിറമായ ഒരു കീരി അവിടെവന്നു ആ തീർഥത്തിൽ കിടന്നുരുണ്ടു. എന്നിട്ടു പറഞ്ഞു “എന്റെ ശരീരം പകുതി സ്വർണ്ണനിറമായത് മറ്റൊരു ദാനനീരിൽ മുങ്ങിയിട്ടാണ്‌. ഇതു വളരെ വിശിഷ്ടമായ ദാനമായിരുന്നെങ്കിൽ എന്റെ ബാക്കി പാതി ശരീരം കൂടി സ്വർണ്ണ നിറമായേനേ.അതുണ്ടായില്ലല്ലോ.”\par
പല കുടുംബങ്ങളിലും വിവാഹം കഴിച്ചുകൊടുത്ത പെൺകുട്ടികൾ ബന്ധം വേർപെടുത്തി സ്വകുടുംബത്തിലേക്കു തിരികെ വരുന്നതിന്റെ കാരണം കന്യാദാനം ഇത ഓം തത് സത് ആയി നടക്കാത്തതുകൊണ്ടാണ്. സന്തതിയായി ഒരു പെൺകുട്ടിയേ ഉള്ളൂ. സ്നേഹക്കൂടുതൽ കൊണ്ടാകാം ഉദകപൂർവ്വം ദാനം കൊടുത്താലും പിടി വിടുകയില്ല.\par

ദാനം കൊടുക്കുന്നവനും വാങ്ങുന്നവനും തുല്യപ്രാധാന്യമാണെങ്കിലും കൊടുക്കുന്നവന്റെ കൈയാണ്‌ വാങ്ങുന്നവന്റേതിനേക്കാൾ അല്പം ഉയർന്നു നില്ക്കുന്നത്.

