\chapter{അപ്പുവിന്റെ അത്ഭുതം}  
അപ്പു അഛനോടും അമ്മയോടുമൊപ്പം ഒരു വേളി കൂടാൻ നാട്ടിൽ പോയതാണു്. നേരത്തെ പോയാലേ എല്ലാവരുമായി വർത്തമാനം പറയാൻ പറ്റൂ എന്നു പറഞ്ഞു് അമ്മ ഉത്സാഹിച്ചതുകൊണ്ടു് നേരത്തെ അവിടെ എത്തിച്ചേർന്നു . അമ്മ, പറഞ്ഞതുപോലെ ഓരോരുത്തരോടും വർത്തമാനം പറഞ്ഞുകൊണ്ടു് നടന്നു. അഛൻ വേളിച്ചടങ്ങുകളിൽ സഹായിക്കാനും വിളമ്പാനും ക്രിയക്കു കൂടാനും ഒക്കെയായി പോയി. രണ്ടുപേരും അപ്പുവിനെ മറന്നതുപോലെ. അപ്പു ഒറ്റയ്ക്കായി. അപ്പു അലങ്കാരങ്ങളും ഒരുക്കങ്ങളും ഒക്കെ കണ്ട് അങ്ങനെ നില്കുകയായിരുന്നു.പെട്ടെന്നു പ്രായമായ ഒരു മുത്തശ്ശി വന്നു ചിരിച്ചു കൊണ്ടുചോദിച്ചു," എന്താ, ഉണ്ണി എന്നെ അറിയ്വോ ?. അപ്പുവിന്റെ മുത്തശ്ശി എന്റെ ഒരു ഏട്ടത്തിയാണു്" .അപ്പു വെളുക്കെ ചിരിച്ചു. മുത്തശ്ശിയും ചിരിച്ചിട്ടു പോയി. എന്തായാലും അപ്പുവിനു് മുത്തശ്ശിയെ ഇഷ്ടമായി. മുത്തശ്ശി പോയ വഴിയേ  നോക്കി നില്ക്കുകയായിരുന്നു അപ്പു.അപ്പുവിനേക്കാൾ പ്രായം കൂടിയ ഒരു കുട്ടി ഓടിവന്നു ചോദിച്ചു, " എന്താ അപ്പുക്കുട്ടൻ ഒറ്റയ്ക്കായിപ്പോയതു? എന്നെ അറിയുമോ? അപ്പുവിന്റെ അമ്മയും എന്റെ അമ്മയും ഒന്നിച്ചു പഠിച്ചവരാണു് ." അപ്പു ആ ചേച്ചിയെത്തന്നെ നോക്കിക്കൊണ്ടു മിണ്ടാതെ നിന്നു.
" അപ്പുക്കുട്ടാ തൊപ്പിക്കാരാ"  എന്നു പറഞ്ഞു ആ കുട്ടി ഓടിപ്പോയി.ആ കുട്ടിയുടെ ഓട്ടവും പാട്ടും കണ്ടു് അപ്പുവിനു് ചിരി വന്നു. അങ്ങനെ നില്ക്കുമ്പോളതാ രണ്ടുമൂന്നേട്ടന്മാർ വലിയ പത്രാസിൽ എന്തൊക്കെയോ പറഞ്ഞുകൊണ്ടു് വരുന്നു. അതിൽ, കണ്ണട വച്ചു മുടി ചപ്രശ്ശാ എന്നായ ഒരാൾ തന്നെ ചൂണ്ടി പറയുന്നതു് കേട്ടു. " ഇതു നമ്മുടെ കുഞ്ഞുണ്ണ്യേട്ടന്റെ പയ്യൻ അപ്പുവല്ലേ?." എന്നിട്ടു അപ്പുവിനോടായിപ്പറഞ്ഞു," തന്റെ മുത്തശ്ശന്റെ ഓപ്പോളുടെ മകളുടെ  മകനാണു് ഞാൻ" . ഇത്രയും പറഞ്ഞു് അവർ വിട്ടുപോയി. അപ്പുവിനു ഒരു പിടിയും കിട്ടിയില്ല. പിന്നെ ഓർത്തപ്പോഴാണു് രസം. ആ ഏട്ടന്റെ മുത്തശ്ശി തന്റെ മുത്തശ്ശന്റെ ഓപ്പോളാണു്. ചുരുക്കി പറയാവുന്ന കാര്യം എന്തിനിത്ര വളച്ചുകെട്ടിപ്പറയുന്നു? ഇതാണോ പത്രാസ്?
     അങ്ങനെ നില്ക്കുമ്പോൾ അഛനോടൊപ്പം സുമുഖനായ ഒരു ചെറുപ്പക്കാരനും തന്റെ പ്രായമുള്ള ഒരു കുട്ടിയും വന്നു. അഛൻ പറഞ്ഞു," കുഞ്ഞുകൃഷ്ണാ,ഇവനാണു് അപ്പു. അപ്പൂ, ഇതു് കൃഷ്ണൻ നായർ. മുത്തപ്ഫന്റെ മകളുടെ മകൻ. ഇതു് ഇയാളുടെ മകൻ സുരേഷ്. നിന്നെപ്പോലെ മൂന്നാം ക്ലാസ്സിൽ പഠിക്കുന്നു" . അപ്പോൾ " കുഞ്ഞുണ്ണീ, ഒന്നിങ്ങട് വരിക" എന്നു ആരോ വിളിയ്ക്കുന്നതു കേട്ടു അഛൻ പോയി. കൃഷ്ണൻ നായരും പോയി. ഉടൻ തന്നെ അമ്മയുടെ പ്രായമുള്ള ഒരു സ്ത്രീ കല്ല്യാണസ്സമ്മാനവും പ്ലാസ്റ്റിക് കൂട്ടിൽ തൂക്കി കടന്നു വന്നു." അപ്പുക്കുഞ്ഞല്ലേ,ഇതു? ഞാൻ ദേവകി വാരസ്യാർ. അപ്പുവിന്റെ അമ്മയുടെ ഇല്ലത്തിനടുത്താണു്" .ഒരിക്കൽ അമ്മാത്തു പോയപ്പോൾ ഇവരെ കണ്ടതു് അപ്പു ഓർത്തു. ഉടനെ അമ്മ എവിടെനിന്നോ ഓടിവന്നു. അപ്പുവിനെയും കൂട്ടി അച്ഛനെ തിരഞ്ഞു പിടിച്ചു. എന്നിട്ടു പറഞ്ഞു," നമുക്കു ഷാരടിമാഷിന്റെ വീടു വരെ ഒന്നു പോകണ്ടേ. മാഷ് വയ്യാതെ കിടക്കുകയാണല്ലോ. ഇനി എന്നാ ഇതിലേ വരാൻ പറ്റുകാ?"  
" ശരി, ആദ്യത്തെ പന്തി സദ്യ കഴിയുമ്പോഴേക്കും ഇങ്ങെത്തണം" എന്നു പറഞ്ഞു അഛൻ. എല്ലാവരും കൂടി കാറിൽ കയറിപ്പോയി.
    മാഷിന്റെ വീടിന്റെ പടിക്കൽ വണ്ടിയിട്ട് അകത്തേക്കു് നടക്കുമ്പോൾ ഒരു പൈക്കിടാവ് ഓടി വന്നു് അപ്പുവിന്റെ മുമ്പിൽ  കൂടി തുള്ളിച്ചാടിപ്പോയി. അപ്പുവിനു് വളരെ സന്തോഷമായി. നല്ല പൂങ്കുലയുള്ള ഒരു ചെടി കാറ്റിലാടി. പൂങ്കുല അപ്പുവിന്റെ തലയിൽ തൊട്ടുപോയി. ഒരു നായ ഓടിവന്നു് അഛന്റെ അരികിൽ വാലാട്ടിനിന്നു.ആകപ്പാടെ മുത്തശ്ശി പറഞ്ഞുതരാറുള്ള ശാകുന്തളത്തിലെ കണ്വാശ്രമം പോലെയുണ്ട്. അപ്പുവിനു പൈക്കുട്ടിയോടും പൂങ്കുലയോടും നായയോടും ഒക്കെ ഒരു മമതാബന്ധം തോന്നി.
   ഷാരടിമാഷ് മുൻ‌വശത്തുതന്നെ ഒരു ചാരുകസാലയിൽ കിടക്കുന്നുണ്ടായിരുന്നു. എല്ലവരേയും കണ്ടപ്പോൾ മാഷിനു് വളരെ സന്തോഷമായി.അപ്പുവിന്റെ തലയിൽ കൈവച്ച് അനുഗ്രഹിച്ചു. അപ്പുവിന്റെ അഛനേയും അമ്മയെയും അമ്മാവനേയും പഠിപ്പിച്ചിട്ടുള്ള മാഷാണു്. കുശലപ്രശ്നങ്ങൾക്കു ശേഷം കണ്വമഹർഷിയെക്കണ്ട പ്രതീതിയോടെ മടങ്ങിപ്പോന്നു.
       വേളിഹാളിൽ എത്തിയപ്പോഴേക്കും ആൾക്കാർ ഭൂരിഭാഗവും പോയിക്കഴിഞ്ഞിരുന്നു.ഒരു പന്തിയിൽ കുറച്ചുപേർ ഇരിക്കാൻ തുടങ്ങുന്നു. അവരുടെ കൂടെ കൂടി. ഒട്ടും തിരക്കില്ലാതെ സൗകര്യമായി സദ്യയുണ്ടു. ഇനിയും വർത്തമാനം പറയാൻ ആരുമില്ലാതിരുന്നതുകൊണ്ട് ഉടൻ തന്നെ എല്ലാവരും കാറിൽ കയറി തിരികെ പോന്നു. പോരുമ്പോൾ വഴിയിൽ എവിടെ നിന്നോ ഒഴുകിയെത്തിയ ഒരു ഗാനം കേട്ടു.
അറ്റമില്ലാത്തൊരീ ജീവിതപ്പാതയി-
ലൊറ്റയല്ലൊറ്റയല്ലൊറ്റയല്ലാ.
അപ്പു ഓർത്തുപോയി, അഛൻ, അമ്മ, മറ്റുബന്ധുക്കൾ, അവരുടെ ചാർച്ചക്കാർ, സുഹൃത്തുക്കൾ, പഠിപ്പിച്ച മാഷന്മാർ, എന്തിനധികം? തൊടിയിലെ ചെടികൾ, വളർത്തുന്ന ജീവികൾ, ഇവയോടെല്ലാം ഒരു അടുപ്പം തോന്നുന്നു.
അപ്പു അത്ഭുതത്തോടെ അറിയാതെ ചോദിച്ചുപോയി" നമ്മൾക്ക് ഒരുപാട് ആൾക്കാരുണ്ട്. അല്ലേ അഛാ?." 
അതെന്താ അപ്പൂ, അങ്ങനെ ചോദിച്ചത്?." 
എത്ര ആളുകളാ എന്നെ ശ്രദ്ധിക്കുകയും കുശലം പറയുകയും ചെയ്യുന്നത്! മാഷിന്റെ വീട്ടിലെ പൈക്കിടാവും നായയും ചെടികളും പോലും എന്നെ ശ്രദ്ധിച്ചതുപോലെ തോന്നി.
അഛൻ പറഞ്ഞു,, " ഈ പ്രപഞ്ചത്തിലെ എല്ല സൃഷ്ടിജാലങ്ങളും ഒരേ ഈശ്വരന്റെ ചൈതന്യം ഉൾക്കൊള്ളുന്നു. എല്ലാ ജീവജാലങ്ങളും ഒരേകുടുംബത്തിലെ അംഗങ്ങളാണു് .മഹാകവി വള്ളത്തോൾ പാടിയിട്ടില്ലേ,
  " ലോകമേതറവാടു തനിയ്ക്കീചെടികളും പുൽക്കളും പുഴുക്കളും കൂടിത്തൻ കുടുംബക്കാർ" .

