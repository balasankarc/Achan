\chapter{മരതകക്കിങ്ങിണിസ്സൗഗന്ധികസ്സ്വർണമായ്}
\obeylines
മകരമാസത്തിലെ ഒരു തെളിഞ്ഞ പ്രഭാതം.അപ്പുവും അഛനുംകൂടി മുറ്റത്തു അലസമായി ഉലാത്തുന്നു.
``asdfഒരു കവിത ചൊല്ലൂ, അഛാ'' അപ്പു ആവശ്യപ്പെട്ടു.
``വൈലോപ്പിള്ളിയുടെ മാമ്പഴം ആയാലോ?''
``ശരി '' അപ്പു സമ്മതിച്ചു.അപ്പുവിനു വളരെ ഇഷ്ടപ്പെട്ട കവിതയാണ്‌.അച്ഛനും അമ്മയും ചൊല്ലി പല തവണ കേട്ടിട്ടുള്ളതാണ്‌. അതിനാൽ അതിലെ പല വരികളും അപ്പുവിനു കാണാപ്പാഠമാണ്‌.
\hspace{2em}അങ്കണത്തൈമാവിൽനിന്നാദ്യത്തെപ്പഴം വീഴ്കെ  
\hspace{2em}അമ്മതൻ നേത്രത്തിൽനിന്നുതിർന്നൂ ചുടു കണ്ണീർ
അമ്മയുടെ മുഖത്ത് സങ്കടം വരുന്നതു കാണാൻ അപ്പു ‘അമ്മ’ എന്നത് നീട്ടി ഉറക്കെ ചൊല്ലി. അതുകേട്ട് അമ്മ മുറ്റത്തേക്കു വന്നു. അച്ഛനും മകനും കൂടി കവിത ആസ്വദിക്കുകയാണെന്നു കണ്ട് തിരിച്ചു കയറിപ്പോയി.
\hspace{2em}നാലു മാസത്തിൻ മുമ്പിലേറെനാൾ കൊതിച്ചിട്ടി-
\hspace{2em}ബ്ബാലമാകന്ദം പൂവിട്ടുണ്ണികൾ വിരിയവേ
ഇതുചൊല്ലിയ അപ്പു മുറ്റത്തേക്കു പടർന്നു തണൽ വിരിച്ചുനില്ക്കുന്ന ആ വലിയ മാവിലേക്കു നോക്കി. മാവു പൂത്തിട്ടുണ്ടോ, ഉണ്ണി വിരിഞ്ഞോ എന്നൊക്കെ അറിയാൻ.
``മാവു പൂത്തു.അതാ ഒരു പൂങ്കുല ''അപ്പുവിനു പൂത്തിരികത്തിച്ചപോലെ സന്തോഷമായി. മാവിനു ഉയരം ഉള്ളതുകൊണ്ട് പൂങ്കുല ഒടിക്കാൻ പറ്റാത്തതിന്റെ സങ്കടവും വന്നു. രണ്ടുപേരും ആലാപനവും ആസ്വാദനവും തുടർന്നു. അപ്പു അതിനു തക്ക ഭാവങ്ങളും ഗോഷ്ടിയുമൊക്കെ കാണിച്ച് മനോഹരമായി ആസ്വദിച്ചു.
\hspace{2em}``തുംഗമാം മീനച്ചൂടാൽ തൈമാവിൻ മരതക
\hspace{2em}കിങ്ങിണി സൌഗന്ധികസ്വർണമായ്ത്തീരും മുമ്പേ
ഈ വരികൾ അഛൻ ആസ്വദിച്ചു രണ്ടുമൂന്നു പ്രാവശ്യം ചൊല്ലി.എന്നിട്ടു അഛൻ പറഞ്ഞു - 
``കണ്ണിമാങ്ങകൾ ഉണ്ടായി പഴുത്ത മാങ്ങകൾ ആകുന്നത് ഇതിലും മനോഹരമായി വർണിക്കാനാവില്ല. ചുരുങ്ങിയ വാക്കുകളിൽ വിടരുന്ന ഭാവന. ഉണ്ണികണ്ണന്റെ അരയിലെ കിങ്ങിണിയുടെ ഓർമ്മകൾ മാവിലെ കണ്ണിമാങ്ങകൾ. മീനമാസത്തിലെ ചൂടു തട്ടി അതു പഴുത്തു പാകമാവുമ്പോൾ നല്ല സുഗന്ധവും സ്വർണ്ണനിറവും കിട്ടുന്നു. സൌഗന്ധികസ്സ്വർണ്ണം. പോക്കുവയിൽ കൂടി തട്ടിയാൽ എന്തൊരു മനോഹാരിതയായിരിക്കും! ''
``ഹായ്! ഹായ്! ''അപ്പു സന്തോഷം കൊണ്ടു തുള്ളിച്ചാടി.
  കവി പറഞ്ഞ കാര്യം ഒന്നു നേരില്കാണണമെന്ന് അപ്പുവിനുതോന്നി.പിന്നെ രണ്ടുമൂന്നു ദിവസത്തേയ്ക്ക് അപ്പു രാവിലേ തന്നെ എഴുന്നേറ്റ് മാവിലേക്ക് നോക്കാൻ തുടങ്ങി. ഒരുദിവസം നോക്കുമ്പോൾ പൂങ്കുല കണ്ടില്ല. അപ്പുവിനു സങ്കടമായി. ആരെങ്കിലും തല്ലിക്കളഞ്ഞതാണോ? അപ്പുസംശയിച്ചു. മുത്തശ്ശൻ പറഞ്ഞു മഴക്കാർ ഉണ്ടായിരുന്നതുകൊണ്ട് മാമ്പൂ ഉരുകിപ്പോയതാണെന്ന്. മഴക്കാർ ഉള്ള ദിവസങ്ങളിൽ അന്തരീക്ഷത്തിൽ ചൂടു കൂടുതലായിരിക്കും. അങ്ങനെ അത് ഉരുകിപ്പോയതാണ്‌. ചില ദിവസങ്ങളിൽ നല്ല ഉഷ്ണം അനുഭവപ്പെടുമ്പോൾ ഇന്നൊരു മഴയ്ക്കു സാധ്യതയുണ്ടെന്ന് മുത്തശ്ശൻ പറയാറുള്ളത് അപ്പു ഓർത്തു. മഴക്കാറുള്ളപ്പോൾ ചൂടിന്‌ അന്തരീക്ഷത്തിനു പുറത്തേക്കു പോകാൻ സാധിക്കുകയില്ല.
ശീതരാജ്യങ്ങളിൽ സസ്യങ്ങൾ വളർത്തുമ്പോൾ ചില്ലുകൊണ്ട് പന്തലിടാറുണ്ട്. അപ്പോൾചൂടു പുറത്തേക്കു പോവുകയില്ല. ചില്ലു അകത്തേക്കുള്ള പ്രകാശത്തെ കടത്തിവിടും എന്നാൽ ചൂടിനെ പുറത്തേക്കു വിടുകയില്ല. ഒരു ദിവസം കാറിൽ യാത്ര ചെയ്തപ്പോൾ പുറകിലത്തെ ചില്ലിൽ കൂടി വെയിലടിച്ചതും കൂടുതൽ ചൂടനുഭവപ്പെട്ടതും അപ്പു ഓർത്തു.ചെടികൾ വളരുന്ന ഇത്തരം ചില്ലുമേടകൾക്ക് ഹരിതഗൃഹം എന്നാണ്‌പറയുന്നത്. ശീതരാജ്യങ്ങളിൽ ഈ ചൂട് കിട്ടുന്നതുകൊണ്ടാണ്‌ ചെടികളിലെ പച്ചപ്പ് നിലനില്ക്കുന്നത്. അതുകൊണ്ട് ഈ അവസ്ഥയെ ‘ഹരിതഗൃഹപ്രഭാവം’ എന്നു പറയുന്നു. അന്തരീക്ഷത്തിലെ കാർബൺ ഡായോക്സൈഡ്, നീരാവി എന്നിവയ്ക്കും ചൂടിനെ ആഗീരണം ചെയ്യാനുള്ള കഴിവുണ്ട്. അതിനെ മറികടന്ന് ചൂട് പുറത്തേക്ക് പോവുകയില്ല. മഴക്കാർ നീരാവിയാണല്ലോ. ചൂടിനെ വലിച്ചെടുക്കും. ഇങ്ങനെ വലിച്ചെടുക്കുന്ന ചൂടിനെ തിരികെ ഇങ്ങോട്ടു തന്നെ വിടുകയും ചെയ്യും. അതായത് ചൂട് അതിൽത്തട്ടി തിരികെ ഭൂമിയിലേക്ക് വരുന്നു. പാരിസ്ഥിതിക രസതന്ത്രം പഠിക്കുകയാണെങ്കിൽ ഇതിനെപ്പറ്റി കൂടുതൽ പഠിക്കാൻ പറ്റും എന്ന് അച്ഛൻ പറഞ്ഞിട്ടുണ്ട്. പിന്നീട് ക്ലാസ്സുകളും പഠിത്തവുമൊക്കെയായി കുറച്ചു ദിവസത്തേക്ക് അപ്പു മാവിന്റെയും മാമ്പൂവിന്റെയും കാര്യം തന്നെ മറന്നുപോയി.
ഒരുദിവസം രാവിലെ സ്ക്കൂളിൽ പോകാനായി അപ്പു ചോറുണ്ണാൻ തുടങ്ങുമ്പോൾ മുത്തശ്ശി ഓടിവന്ന്, ``അപ്പൂന്‌ ഒരൂട്ടം വേണ്ടേ? നല്ല സ്വാദുണ്ട് '' എന്നുപറഞ്ഞൊരു ചെറിയ ഭരണിയിൽനിന്ന് സ്വല്പം നിറം മങ്ങിയ കുഞ്ഞുമാങ്ങാക്കഷണങ്ങൾ വെള്ളത്തോടുകൂടി വിളമ്പിത്തന്നു.
``ഹായ്! നല്ലസ്വാദ്! ഇതെന്താ? മുത്തശ്ശീ, എനിക്കു സ്കൂളിലെ ചോറ്റുപാത്രത്തിലും ഇതിട്ടു തരണേ. ''
``രണ്ടീസം മുമ്പ് മുറ്റത്തെ മാവിഞ്ചുവട്ടിൽനിന്ന് കിട്ടിയ കണ്ണിമാങ്ങയാണ്‌. കണ്ണിമാങ്ങ ഉപ്പു തിരുമ്മി വച്ചാൽ നല്ല രുചിയാണ്‌ '' എന്നുപറഞ്ഞ് അപ്പുവിന്റെ ചോറ്റുപാത്രത്തിലും വച്ചുകൊടുത്തു.
പിറ്റെദിവസം സ്ക്കൂളില്ലായിരുന്നു. അപ്പു മാവിഞ്ചുവട്ടിൽ പോയിനോക്കി. അവിടവിടെയായി കണ്ണിമാങ്ങകൾ വീണുകിടപ്പുണ്ടായിരുന്നു. അപ്പു അതൊക്കെ പെറുക്കിയെടുത്തിട്ട് മുകളിലേക്കു നോക്കി ഇളം വെയിൽ തട്ടിത്തിളങ്ങുന്ന ‘മരതകകിങ്ങിണികൾ’. കുറച്ചൊക്കെ കൊഴിഞ്ഞു താഴെ വീണിട്ടുണ്ടെങ്കിലും മാവിൽ കണ്ണിമാങ്ങകളുടെ കുലകൾതന്നെയുണ്ട്. അപ്പു അതു മുത്തശ്ശനെയും മുത്തശ്ശിയെയും അച്ഛനെയും അമ്മയെയും വിളിച്ചു കാണിച്ചുകൊടുത്തു.
കുറച്ചു ദിവസത്തേക്കുകൂടി മുത്തശ്ശി കണ്ണിമാങ്ങാ ഉപ്പുതിരുമ്മിയത് കൊടുത്തു. വാർഷികപ്പരീക്ഷ ആയതിനാൽ അപ്പു പിന്നീട് ഒന്നിനെപ്പറ്റിയും ഓർത്തുമില്ല ചിന്തിച്ചതുമില്ല.
ഒരുദിവസം പരീക്ഷകൾ തീർന്ന ആശ്വാസത്തിൽ ഇല്ലത്തെത്തിയ അപ്പുവിനു മുത്തശ്ശി പൂളി മുറിച്ച മാമ്പഴക്കഷണങ്ങൾ ഒരു പ്ലേറ്റിൽ വച്ചുകൊടുത്തു. ``നല്ല ശർക്കരമാമ്പഴം. മുറ്റത്തെ മാവിൽനിന്നും വീണുകിട്ടിയതാ.ഇനി അഞ്ചാറെണ്ണം കൂടിയുണ്ട് ''മുത്തശ്ശി പറഞ്ഞു.
അപ്പു ഒരു കഷണം കഴിച്ചുനോക്കി. ഹായ് !നല്ല മധുരം.പെട്ടെന്നുതന്നെ മുഴുവൻ കഴിച്ചുതീർത്തു.ഇനിയെവിടെയാ മുത്തശ്ശീ എന്നു ചോദിച്ചുകൊണ്ട് ഓടിപ്പോയി രണ്ടെണ്ണം എടുത്തു കൊണ്ടുവന്നു. ഒരെണ്ണം മുത്തശ്ശനു കൊടുത്തു. ഊണിനു പിഴിഞ്ഞുകൂട്ടാം എന്നു പറഞ്ഞു മുത്തശ്ശൻ അതു സൂക്ഷിച്ചു വച്ചു. ഇനിയുള്ള ഒരെണ്ണം കൈയിൽ പിടിച്ചു അപ്പു അതിന്റെ ഭംഗി നോക്കി.നല്ല സ്വർണ്ണനിറം. ഇനിയും മാമ്പഴങ്ങൾ മാവിലുണ്ടോ എന്നറിയാൻ മുറ്റത്തുചെന്നു മാവിലേക്കുനോക്കി. ഹായ്! ധാരാളം മാമ്പഴങ്ങൾ. സ്വതവേ സ്വർണ്ണനിറമുള്ള അവ അന്തിവെയിലിൽ തിളങ്ങുന്നു. ഇതുകണ്ടാൽ വൈലോപ്പിള്ളിയല്ല ആരും കവിത എഴുതിപ്പോകും.
അച്ഛൻ വന്നപ്പോൾ അപ്പു വിശേഷങ്ങളൊക്കെ പറഞ്ഞു. പരീക്ഷ തീർന്ന കാര്യത്തേക്കാൾ മരതകക്കിങ്ങിണി സൌഗന്ധികസ്സ്വർണ്ണമായതിനേപ്പറ്റി പറയാനായിരുന്നു കൂടുതൽ ഉത്സാഹം.
അഛൻ ചോദിച്ചു, ``അപ്പു കണ്ടുവോ? നല്ല ഭംഗിയില്ലേ? വൈലോപ്പിള്ളിയേപ്പോലെ കവിത വല്ലതും തോന്നിയോ?''
``കവിതയൊന്നും തോന്നിയില്ല.എന്നാലും ഒരു സംശയം. ഉണ്ണിമാങ്ങായെങ്ങനെയാ പഴുത്ത മാങ്ങായായത്?''
``വളരുക എന്നത് ഒരു പ്രകൃതിനിയമമല്ലേ, ഉണ്ണീ?'' മുത്തശ്ശൻ
 ശരിയാണ്‌. അപ്പു ഓർത്തു. താൻ ചറുതായിരുന്നപ്പോഴത്തെ കഥകൾ അമ്മ പറയാറുണ്ട്. അഛനും അമ്മയും ചെറുതായിരുന്നപ്പോഴത്തെ കഥകൾ മുത്തശ്ശിമാർ പറയറുണ്ട്. മുത്തശ്ശൻ ചെറുതായിരുന്നപ്പോൾ കാണിച്ച കുസൃതിത്തരങ്ങൾ മുത്തശ്ശൻ തന്നെ പറയാറുണ്ട്. എല്ലാവരും വളർന്നു വലുതായവരാണ്‌. ശരിയാണ്‌.  മാങ്ങ വളർന്നോട്ടെ. പക്ഷെ പഴുക്കുന്നതെന്തിനാ?എങ്ങനെയാ?
അഛൻ ചോദിച്ചു`` മാവുണ്ടാകുന്നതെങ്ങനെയാ? ''
``പ്ലാസ്റ്റിക് കൂട്ടിൽ ''അപ്പുവിനു സംശയമില്ലായിരുന്നു. ഇന്നാള് നല്ലയിനം മാവാണെന്നു പറഞ്ഞച്ഛൻ കൊണ്ടുവന്ന മാവിൻ തൈ ഒരു കറുത്ത പ്ലാസ്റ്റിക് കൂട്ടിലായിരുന്നു.
അഛൻ പറഞ്ഞു,``പ്ലാസ്റ്റിക് കൂട് വെറുതെ വച്ചിരുന്നാൽ മാവിൻ തൈ ഉണ്ടാവുകയില്ല. അതിൽ മാങ്ങാണ്ടി കുഴിച്ചിടണം. താഴത്തെ പറമ്പിന്റെ മൂലയ്ക്ക് ആരോ കളഞ്ഞിരുന്ന കുറെ മാങ്ങാണ്ടികൾ കൂട്ടം കൂടി മുളച്ചുനില്ക്കുന്നത് നമ്മൾ കണ്ടത് അപ്പു ഓർക്കുന്നില്ലേ? ''
``ശരിയാ. പക്ഷേ അതിൽ ഒന്നുപോലും പിടിച്ചില്ലല്ലോ. ''
``മാങ്ങാണ്ടികളെല്ലാം മാവിഞ്ചുവട്ടിൽ തന്നെ കൂടിക്കിടന്നാൽ എല്ലാത്തിനും വളരാൻ ഇടവും സൌകര്യവും കിട്ടുകയില്ല. അതുകൊണ്ട് മാങ്ങാണ്ടികൾ മറ്റെവിടെയെങ്കിലും ഒക്കെ എത്തിക്കണം. അതിനു മനുഷ്യരെക്കാൾ നല്ലത് പക്ഷിമൃഗാദികളാണ്‌. കിളികൾ മാമ്പഴം കൊത്തിക്കൊണ്ടുപോകുന്നത് അപ്പു കണ്ടിട്ടില്ലേ? അവ മാമ്പഴം പല സ്ഥലങ്ങളിലും കൊണ്ടുവച്ചു തിന്നിട്ട് അണ്ടി അവിടെ ഉപേക്ഷിക്കും. അതു അവിടെക്കിടന്നു വളരും. പക്ഷികളെ ആകർഷിക്കാൻ വേണ്ടിയാണ്‌ ആകർഷകമായ നിറവും മണവും പഴങ്ങൾക്കു കിട്ടിയിരിക്കുന്നത്. അങ്ങനെ വിത്തുകളെ വിതരണം ചെയ്യുന്ന പദ്ധതിയും പ്രകൃതി തന്നെ ആവിഷ്കരിച്ചിരിക്കുന്നു. ''
``ഇപ്പോഴും പച്ചനിറം സ്വർണ്ണനിറമാവുന്ന മാജിക് എന്താണെന്നു പറഞ്ഞില്ല. ''
``ഇലകൾക്കും പച്ചമാങ്ങാക്കും പച്ചനിറം കിട്ടുന്നതു ഹരിതകം(chlorophyll)എന്ന യൌഗികം ഉള്ളതുകൊണ്ടാണ്‌. മാങ്ങാ പഴുക്കാൻ തുടങ്ങുമ്പോൾ ഹരിതകം നഷ്ടപ്പെടാൻ തുടങ്ങുന്നു. പച്ചനിറം കുറയുന്നു.അതോടൊപ്പം ഫലം മാംസളമാവുകയും തൊലിക്കു മറ്റു വർണ്ണങ്ങൾ ഉണ്ടാവുകയും ചെയ്യുന്നു. പഴങ്ങൾക്ക് രണ്ടുതരത്തിലുള്ള വർണ്ണകങ്ങളാണുള്ളത്. കരോട്ടിനോയ്ഡുകളും ആന്തോസൈനീനുകളും. മാമ്പഴത്തിനുണ്ടാവുന്ന മഞ്ഞ ചുവപ്പ്മുതലായ നിറങ്ങൾ കരൊട്ടിനോയിഡ് ഗ്രൂപ്പിൽ പെട്ടവയാണ് ആപ്പിളിന്റെയും മുന്തിരിയുടെയും നിറങ്ങൾ ആന്തോസൈനിൻ വകുപ്പിൽ പെട്ടവയാണ്‌. ഇവയെയൊക്കെ രാസപരമായി തിരിച്ചറിഞ്ഞിട്ടുണ്ട്. കാർബണിക രസതന്ത്രം കൂടുതൽ പഠിക്കുമ്പോൾ ഇതൊക്കെ മനസ്സിലാകും.``
``മാമ്പഴത്തിനു നല്ല മണവുമുണ്ടല്ലോ. ''
``അതെ,പഴങ്ങൾക്കു നല്ല വാസന ഉണ്ടാക്കുന്ന യൌഗികങ്ങളും പഴുക്കുന്നതോടൊപ്പം ഉണ്ടാവും. ഇവയെല്ലാം എസ്റ്റർ വിഭാഗത്തിൽ പെട്ടവയാണ്‌. ''
``പച്ചമാങ്ങായ്ക്കു പുളിയും പഴുത്തമാങ്ങാക്ക് മധുരവുമാണല്ലോ.പുളി മധുരമാകുന്ന മാജിക്കെന്താണ് ? ''
``പച്ച മാങ്ങായിൽ പലതരം അമ്ലങ്ങൾ (ആസിഡ്) ഉണ്ട്. ഓക്സാലിക്കാസിഡ്, സിട്രിക്കാസിഡ്, സക്സിനിക്കാസിഡ്, മാലിക്കാസിഡ് എന്നിവയാണവ.ആസിഡുകൾക്കു പുളിപ്പുണ്ട്.മാങ്ങാ പഴുക്കുമ്പോൾ കോശങ്ങളിലെ മൈറ്റോകോൺട്രിയയിൽ വച്ച് അവ ഓക്സീകരിക്കപ്പെട്ട് കാർബൺ ഡൈ ഓക്സൈഡും ജലവും ആയിത്തീരുന്നു. ആസിഡ് നശിക്കുന്നതോടെ പുളിയും നഷ്ടപ്പെടുന്നു. അതോടൊപ്പം പച്ചമാങ്ങായിലുള്ള കാർബോഹൈഡ്രേറ്റുകൾ പഞ്ചസാരയായി മാറുന്നു. ഗ്ലൂക്കോസ്, ഫ്രക്റ്റോസ്, സൂക്രോസ് മുതലായവ. അങ്ങനെ മധുരം ഉണ്ടാകുന്നു.
ഈ പ്രക്രിയകൾ എളുപ്പത്തിൽ നടക്കാൻ സഹായിക്കുന്ന ചില രാസപദാർത്ഥങ്ങളുണ്ട്. വളർച്ചയെ സഹായിക്കുന്ന ഹോർമ്മോണുകളാണവ. പ്രധാനമായും എഥിലിൻ, അസെറ്റിലിൻ മുതലായ വാതകങ്ങൾ. പഴുത്തു തുടങ്ങുമ്പോൾ ഇവയുടെ ഉത്പാദനവും കൂടുന്നു. അതുകൊണ്ട് ഒരെണ്ണം പഴുത്താൽ സമീപത്തുള്ളവയും പെട്ടെന്ന്പഴുക്കാൻ തുടങ്ങും. മാവിൽനിന്നും പറിച്ചെടുത്ത മാങ്ങകളെ വേഗത്തിൽ പഴുപ്പിക്കണമെങ്കിൽ ഈ വാതകങ്ങളെ കൃത്രിമമായി നല്കിയാൽ മതി. പച്ചമാങ്ങകളെ എളുപ്പത്തിൽ പഴുപ്പിക്കാൻ കച്ചവടക്കാർ ചെയ്യുന്ന ഒരു വിദ്യയാണ്‌ കാൽസ്യം കാർബൈഡ് തരികൾ വിതറുകയെന്നത്. ഈർപ്പവുമായിച്ചേർന്നു ഈ തരികൾ അസെറ്റിലിൻ വാതകമുണ്ടാക്കുന്നു. ഇതു പഴുക്കൽ പ്രക്രിയയെ സഹായിക്കുന്നു, ത്വരിതപ്പെടുത്തുന്നു. അതുകൊണ്ടാണ്‌ പുറമേനിന്നു കച്ചവടക്കർ കൊണ്ടുവരുന്ന മാമ്പഴങ്ങളെല്ലം ഒരുപോലെ നല്ലനിറത്തിൽ പഴുത്തിരിക്കുന്നത്``.
``പുറമെനിന്നുള്ള മാമ്പഴം അധികം കഴിക്കരുതെന്ന് അമ്മ പറഞ്ഞല്ലോ, അതെന്താ? ''
``കച്ചവടക്കാർ പഴുപ്പിക്കാനുപയോഗിക്കുന്ന കാൽസ്യം കാർബൈഡിൽ അഴ്സെനിക് പോലുള്ള വിഷവസ്തുക്കളും ഉണ്ട്. ഇവ മാമ്പഴത്തിൽ കടന്നുകൂടും. ഇതു പല തരം അസുഖങ്ങൾക്കും കാരണമാകും. ഈ വിഷവസ്തുക്കൾ കൂടുതലായും മാമ്പഴത്തിന്റെ തൊലിയിലാണ്‌ ആഗിരണം ചെയ്യപ്പെടുക. അതുകൊണ്ട് പുറമേനിന്നു വാങ്ങുന്ന മാമ്പഴത്തിന്റെ തൊലി ചെത്തിയേ കഴിക്കാവൂ. എന്നാൽ നമ്മുടെ തൊടിയിലെ മാമ്പഴമാണെങ്കിൽ തൊലി കളയാതെ കഴിക്കണം. അതിൽ ധാരാളം പോഷകങ്ങൾ ഉണ്ട്, നാരുകൾ ഉണ്ട്,മധുരത്തെ നിയന്ത്രിക്കുന്ന വസ്തുക്കളുണ്ട് ''.
ഇങ്ങനെ പറഞ്ഞുകൊണ്ടിരിക്കുമ്പോൾ ഒരു നല്ല കാറ്റു വീശി. പകൽസമയം നല്ല ചൂടുണ്ടായിരുന്നതുകൊണ്ട് ഒരു സുഖം തോന്നി. പൊടു പൊടോ എന്നു വീഴുന്ന ശബ്ദം കേട്ടു അമ്മ പറഞ്ഞു
``ധാരാളം മാമ്പഴം വീണിട്ടുണ്ടാകും പെറുക്കിക്കൊണ്ടുവന്നാൽ നല്ല മാമ്പഴപ്പുളിശ്ശേരി വച്ചുതരാം ''.
മുറ്റത്തു ചെന്നുനോക്കിയപ്പോൾ പറഞ്ഞതുപോലെ ധാരാളം മാമ്പഴം വീണിരിക്കുന്നു. എല്ലാം കൂടിപെറുക്കിയപ്പോൾ ഒരു കുട്ടനിറയെ കിട്ടി.
``അപ്പൂന്‌ സന്തോഷമായില്ലേ? ''മുത്തശ്ശൻ ചോദിച്ചു.
``ആയി,പക്ഷെ ഇതെല്ലാം ഒരുമിച്ച് കഴിച്ചു തീർക്കാൻ പറ്റുകില്ലല്ലൊ.കുറച്ചുദിവസം കഴിയുമ്പോൾ ഇതെല്ലാം ചീഞ്ഞു പോകുകയും ചെയ്യുമല്ലോ. കാറ്റുവരുമ്പോൾ മാവിലെ മാമ്പഴം മുഴുവൻ വീണു പോവുകയുംചെയ്യും. ഇതൊക്കെയോർക്കുമ്പോൾ സന്തോഷം കുറഞ്ഞുപോകുന്നു. കൂട്ടുകാർക്കെല്ലാം നല്ല മാമ്പഴം കൊടുക്കാം,അല്ലേ മുത്തശ്ശാ?അപ്പോൾ നല്ല സന്തോഷമായിരിക്കും ''.
  മുറ്റത്തു മാവില്ലാത്തവരും ഫ്ലാറ്റിൽ താമസിക്കുന്നവരുമായി ധാരാളം കൂട്ടുകാർ അപ്പുവിനുണ്ട്. അവർക്കൊന്നും പ്രകൃതിയുടെ ഈ വരദാനം ,മരതക കിങ്ങിണി സൌഗന്ധിക സ്വർണ്ണമാകുന്നതും ചക്കരമാമ്പഴത്തിന്റെ ശരിയായ രുചിയും അനുഭവിക്കാൻ കഴിയുകയില്ലല്ലോ എന്നോർത്തപ്പോളപ്പുവിനു സങ്കടം വന്നു.
മുത്തശ്ശൻ പറഞ്ഞു ``എന്റെ ചെറുപ്പകാലത്തു എല്ലാവരുടെയും വീടിന്റെ തൊടിയിൽ ധാരാളം മാവും മാമ്പഴവും ഉണ്ടായിരുന്നു. ആർക്കും പ്രത്യേകിച്ചു കൊടുക്കേണ്ട അവസ്ഥ ഉണ്ടായിരുന്നില്ല. കുട്ടികളൊക്കെ കാറ്റു വീശുമ്പോൾ എല്ലാ മാവിന്റെയും ചുവട്ടിൽ ചെന്നു പെറുക്കുകയുംചെയ്യും. അതിനു മടിയോ വിലക്കുകളോ ഉണ്ടായിരുന്നില്ല. പ്രകൃതിയിലെ പഴങ്ങൾ എല്ലാക്കുട്ടികൾക്കുമവകാശപ്പെട്ടതാണ്‌. മിച്ചം വരുന്ന മാമ്പഴം പൂളി വെയിലത്തുവച്ച് ഉണക്കിസ്സൂക്ഷിയ്ക്കുമായിരുന്നു. അതുപോലെ മാമ്പഴം പിഴിഞ്ഞുകിട്ടുന്ന ദശയും നീരും ഉണക്കിസ്സൂക്ഷിയ്ക്കും. ഇതിനു മാമ്പഴത്തിര എന്നാണ്‌ പറയുന്നത്. ഇതു രണ്ടും മാമ്പഴമില്ലാത്ത വർഷക്കാലത്ത് കറിവെയ്ക്കാൻ ഉപയോഗിക്കും. ഇപ്പോൾ എല്ലാക്കാലത്തും മാങ്ങ കിട്ടുമല്ലോ. ഹോർമ്മോണുകൾ കുത്തിവച്ചു ഏതുകാലത്തും ഫലങ്ങളും പഴങ്ങളും ഉണ്ടാക്കാനുള്ള വിദ്യയുണ്ട്. ഏതു കാലവസ്ഥയിലും വളരാനും പൂക്കാനുംകായ്ക്കാനും കഴിവുള്ള ചെടികളെ ബയോടെക്നോളജി വഴിയും ജെനെറ്റിക് എഞ്ജിനീയറിങ്ങ് വഴിയും ഉണ്ടാക്കിയെടുക്കാനാവും.ശാസ്ത്രത്തിന്റെയൊരു വളർച്ച!``.
\hspace{2em}``നമ്മൾ വിളിപ്പൂ ലോകത്തെസ്സഖി,നമ്മൾവിളിപ്പൂകാലത്തെ ''
അച്ഛൻ അക്കിത്തത്തിന്റെ വരികൾമൂളി. ``മാമ്പഴപ്പുളിശ്ശേരിയായി.ഇനി എല്ലാവർക്കും ഊണുകഴിക്കാം '' എന്ന് അമ്മ വിളിച്ചുപറഞ്ഞതുകേട്ട് എല്ലാവരും എഴുന്നേറ്റു. 
